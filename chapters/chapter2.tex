\chapter{Literature Review}
\label{chap:lr}
\chaptermark{}
The purpose of this chapter is to explore existing approaches on Radio Resource Management (RRM) in IEEE 802.11 networks, including surveying what is proposed as a part of the 802.11 standard itself, what research has been done, and what is offered by existing commercial solutions. This chapter is organized as follows:
\begin{itemize}
    \item Section \ref{chap:lr:sec:80211_overview} briefly reviews the fundamentals of Radio Frequency (RF) communications, and IEEE 802.11 standard for wireless LAN;
    \item Section \ref{chap:lr:sec:rrm_80211} provides an overview how RRM is facilitated within the IEEE 802.11 standard;
    \item Section \ref{chap:lr:sec:prev_works} provides a synthesis on previous research in RRM;
    \item Section \ref{chap:lr:sec:prop_rrm} provides an overview of proprietary RRM solutions from major vendors;
    \item Section \ref{chap:lr:sec:conclusion} summarizes the chapter.
\end{itemize}
Table \ref{appx:table:defs_table} contains the list of definitions used in this chapter.

\section{Overview of IEEE 802.11 standard}
\label{chap:lr:sec:80211_overview}

Throughout this thesis, I refer to domain-specific terms whose definitions are given in Table \ref{appx:table:defs_table}. In this chapter, I briefly review the IEEE 802.11 standard and highlight the aspects most relevant to this study.


\subsection{Transmission medium}

The primary medium for communications in IEEE 802.11 is electromagnetic (EM) radio frequency (RF) waves operating within the microwave range \cite{colemanCWNACertifiedWireless2021}.

\subsection{Frequency band}
Main frequency bands are: the 2.4 GHz band introduced by 802.11b, the 5 GHz band introduced by 802.11a, and the 6 GHz band, introduced by 802.11ax. This study focuses on the first two bands as used the most frequently. In the following I describe those bands in more detail.

Most of the 802.11 amendments, including b,g,n, and partially ax, operate at the unlicensed 2.4 GHz ISM (Industrial, Scientific, Medical) RF band \cite{tanenbaumComputerNetworks2020, colemanCWNACertifiedWireless2021}. However, using an unlicensed frequency band introduces multiple challenges: the radio spectrum becomes congested with non-802.11 sources, such as microwave ovens, Bluetooth Personal Area Networks (PAN), cordless phones, etc. \cite{tanenbaumComputerNetworks2020, colemanCWNACertifiedWireless2021}. Moreover, 2.4 GHz signals can propagate through solid obstructions such as walls, doors, and windows better than signals operating on higher frequencies \cite{colemanCWNACertifiedWireless2021}. This property can provide better coverage and signal quality for clients, although it can cause interference for neighboring WLANs, which will in turn lead to degradation of signal quality.

The 2.4 GHz ISM band is split into 14 channels. Depending on local regulations, the number of possible channels can vary, but in general channels 1-11 are available in all regions. Assuming a 20 MHz channel width, each channel is characterized by its \textit{center frequency}, $\pm$ 10 MHz, with a 5 MHz space between the centers of two adjacent channels, that is, the channels \textit{ overlap}. Channel 1 has a central frequency of 2.412 GHz, Channel 14 — 2.484 GHz. Thus, for channels to be non-overlapping, they must have at least 5 channels or 25 MHz space between the centers. Such non-overlapping channels are 1, 6, 11, with central frequencies 2.412, 2.437, and 2.462 MHz, respectively.

The 5 GHz U-NII (Unlicensed National Information Infrastructure) series of bands is used by the 802.11a/802.11ac/802.11ax amendments. Unlike the 2.4 GHz band, the channels do not overlap. At the end of each band there is a \textit{guard band} as an additional measure to avoid interference. Combined, the bands U-NII-1, U-NII-1, U-NII-2A, U-NII-2C, U-NII-3 provide twenty-five 20 MHz or twelve 40 MHz non-overlapping channels \cite{colemanCWNACertifiedWireless2021}. However, some channels may not be available in different regions, as this band can also be used by military and weather radars. The first channel from the U-NII-1 band has number 36.


\subsection{Signal quality and its metrics}

Thus, the presence of physical obstructions, background noise, and interference from other access points urges us to explore possible measurements and metrics for wireless signal quality. In the following, I briefly describe the most widely used quantities.

A measure widely used in RF engineering and employed by Wi-Fi vendors is \textbf{Signal-to-Noise Ratio (SNR)}, which is defined as a ratio between the received signal power and the power of background noise:
\begin{equation}
    \label{formula:snr}
    SNR = \frac{P_{signal}}{P_{noise}}
\end{equation}
    Since SNR is essentially a difference in power, which is measured in Watts, in practice it is measured in a relative unit on a logarithmic scale called \textbf{decibel (dB)} [1,2]:

\begin{equation}
    \label{formula:snr_db}
    {SNR}_{dB} = 10\log_{10}\frac{P_{signal}}{P_{noise}}
\end{equation}

In recent years, \textbf{Signal-to-Interference-Plus-Noise ratio (SINR)} measurement have become a more widespread measurement of wireless networks\' signal quality. Similarly, it is defined as:

\begin{equation}
    \label{formula:sinr}
    SINR = \frac{P_{signal}}{P_{noise} + P_{interf}}
\end{equation}

    where $P_{signal}$ is the power of the signal of interest, $P_{interf}$ is the power of interference signals, and $P_{noise}$ is the power of background noise.
    By considering interference from other 802.11 devices, which is typically a dynamic quantity that changes rapidly over time unlike background noise, SINR describes the EM spectrum situation more accurately.

Received Signal Strength Indicator (RSSI) is a relative measure of signal strength in the range from 0 to 255, where 0 is the weakest signal a receiver can sense. The exact correspondence between RSSI and received signal power is implementation-specific and is left to the hardware manufacturers \cite{colemanCWNACertifiedWireless2021}.

\subsection{Radio Resource Management}
\label{chap:intro:sec:rrm}
The scarcity of available frequency bands in the time of growing demand for wireless connectivity has led to the development of methods called \textit{spectrum management} or \textit{radio resource management (RRM)}. Most of the research on RRM is focused on cellular networks, where the coverage area of base stations spans several kilometers, and the number of clients for one station can reach several thousands, so proper spectrum management is vital for operation of cells. However, from a physical layer perspective, the radio situation in 802.11 networks is similar. As described in \cite{zanderRadioResourceManagement1997}, given a wireless network with a set of access points $\boldsymbol{B}$, which can communicate over a set of channels $\boldsymbol{C}$, with a maximum transmit power of $P_{max}$, establishing a radio link between a client device and an access point requires the wireless infrastructure to assign:
\begin{enumerate}
    \item An access point $b \in \boldsymbol{B}$;
    \item A frequency channel $c \in \boldsymbol{C}$;
    \item A transmission power level $ p \leq P_{max}$.
\end{enumerate}
Obviously, channel and transmission power are \textit{global} for a given access point in a sense that all its clients will have to adjust their parameters correspondingly: switch the operating channel and deal with the new received signal strength from their AP.
In Wi-Fi, the first requirement is usually managed by the client itself: the user chooses the SSID they wish to use, and in the case if multiple APs have the same SSID, a client device associates with AP having the strongest signal available. Later, a client can switch to another access point within the same extended service set via \textit{roaming} methods, such as \textit{Fast Basic Service Set (BSS) Transition} defined in 802.11r \cite{ieee80211r}. The roaming decision is ultimately made by a client device, which sends a reassociation request to start the roaming process \cite{colemanCWNACertifiedWireless2021}. However, the access point can force a client to find another access point by sending a deauthentication frame or moving to another channel without notifying.
The second and third requirements are part of the current AP configuration and subject to change. A client discovers the current operating channel of the APs by tuning each available channel in succession, while transmission power can only be estimated by measuring the received signal strength.
Thus, \textbf{the goal of a radio resource allocation algorithm is to optimize spectrum usage within a WLAN by assigning an operating channel and a transmission power level to each access point in a way that maximizes the overall performance of the network.}

As will be shown in Section \ref{chap:lr:sec:rrm_80211}, the 802.11 standard does not provide any algorithms for channel and transmission power assignment; however, some amendments introduce methods for measurement, signaling and radio adjustment that can be used for RRM purposes.

Note that related research and commercial solutions introduce many similar terms for the same channel change procedure that can use different algorithms and slightly vary according to the specifics of their application: \textit{Frequency Selection}, \textit{Frequency Planning}, \textit{Channel Selection}, \textit{Channel Planning}, \textit{Channel Assignment}, etc. 
% Adjustment of transmission power is usually
In this study, we use these terms interchangeably.

\section {Radio Resource Management in IEEE 802.11}
\label{chap:lr:sec:rrm_80211}
As described in Section \ref{chap:intro:sec:rrm}, an RRM algorithm seeks to improve network capacity by optimizing spectrum usage through the adjustment of two parameters: the channel frequency and the transmission power of each access point. The very need for an RRM algorithm comes from a scenario known as \textit{Overlapping Basic Service Sets (OBSSs)} in the IEEE 802.11 standard, when there are two basic service sets operating on the same channel within reach of each other. This scenario leads to co-channel interference, which degrades network performance. In this section, I provide an overview of the methods and techniques provided by the IEEE 802.11 standard that can assist in RRM.


\subsection {IEEE 802.11h}
\label{chap:lr:sec:80211h}
To comply with the legal requirements on transmissions at 5 GHz, the 802.11h-2003 \cite{ieee80211h} amendment (\textit{Spectrum and Transmit Power Management Extensions}) was introduced. The goal of this amendment is to prevent legacy 802.11a 5GHz APs from interfering with radars. 802.11h introduces \cite{konsgenSpectrumManagementAlgorithms2010} spectrum management methods, namely, \textit{Dynamic Frequency Selection (DFS)}, which facilitates automatic change of AP's operating frequency, and \textit{Transmit Power Control (TPC)}, which adjusts AP's transmission power.
However, these methods are not designed to implement RRM for the purpose of enhancing network capacity; instead, they focus on avoiding interference with radars functioning in the 5 GHz spectrum. 

802.11h describes procedures for: quieting the channel to detect presence of a radar, switching to another channel if a radar is detected, and notifying the client stations (STAs) about channel switch. However, the description of a radar detection procedure is beyond the scope of the 802.11h standard \cite{ieee80211h}.
However, the frame types and measurement reporting mechanisms introduced in 802.11h can be considered as a foundation for further research on RRM algorithms.


% \subsubsection{Transmit Power Control}

\subsection {IEEE 802.11k}
\label{chap:lr:sec:80211k}
The amendment 802.11k \cite{roamingieee80211k} (\textit{Radio Resource Measurement}, not to be confused with \textit{Radio Resource Management}) improves the performance of roaming. Roaming is a process of the station moving from one access point to another \cite{colemanCWNACertifiedWireless2021}. The amendment 802.11k improves roaming by allowing stations to request from access points various reports, such as \textit{neighbor reports} for discovering possible access points to which STA can roam, \textit{ link measurements} to estimate how well AP can hear the station, etc. This allows stations to reduce power consumption, speed up roaming and decrease power consumption and airtime usage spent on sending probe requests to each channel when trying to find another AP to roam.
However, this amendment is also not aimed at optimizing network capacity, but rather at improving roaming performance.

\subsection {IEEE 802.11ax}
\label{chap:lr:sec:80211ax}

The amendment 802.11ax \cite{ieee80211ax} (\textit{High Efficiency WLAN}) introduces a means to address the OBSS problem. The amendment introduces \textit{Basic Service Set (BSS) Coloring}, a method to distinguish between different BSSs operating on the same channel. It works as follows: each BSS is assigned with \textit{color}, a 6-bit value spanning from 1 to 63, that resides in the PHY header. Devices belonging to the same BSS check this color when demodulating transmissions (that is, check if this frame is an \textit{intra-BSS}). If the frame has an incorrect color (\textit{inter-BSS} frame), the device stops further demodulation, thus conserving processing resources. \cite{CiscoCatalyst9800}.
If an inter-BSS frame is found to use the same color, AP switches to another color.
One can see that such a solution completely relies on the CSMA/CA media access control (MAC) mechanism and is not able to address the hidden terminal problem. That is, if the AP serving the overlapping BSS is a hidden terminal, the signal will be disrupted near the destination, so the receiving station will not be able to demodulate the signal to inspect the color tag.
Thus, this technique does not solve the problem of co-channel interference and does not provide channel planning, so this cannot be considered as a complete RRM solution.



\section {Previous Works}
\label{chap:lr:sec:prev_works}
Interference from other access points (APs) poses a serious obstacle \cite{suiHowBadAre2015} to provide an acceptable quality of service in large 802.11 WLAN deployments. While the 802.11 carrier-sense MAC protocol is designed to withstand interference, however, interference reduces available airtime and causes frame losses. Thus, network capacity and performance tend to drastically degrade \cite{levantiCAPWAPCompliantSolutionRadio2007}.% For interference between cells within a single WLAN, this problem can in principle be solved by proper \textbf{site surveying}, i.e., planning of geographical placement of the cells. 

In Wi-Fi WLANs, interference mostly originates from \textit{rogue access points} (RAPs), which are operated by third parties and, in general, are not under the control of WLAN administrators. According to \cite{suiHowBadAre2015}, interference from rogue APs can introduce up to 50\% delays in a WLAN. Moreover, the prevalent amount (more than 70\%) of rogue APs are stationary \cite{suiHowBadAre2015}, so their radio presence can be considered as a constant factor in the WLAN. Note that \textit{rogue} does not imply that these APs are malicious or pose threats other than congesting channels and occuping airtime as a result of their legitimate operation.

In this light, attempts to improve spectrum management through channel assignment and transmit power control adjustment algorithms encompass research on RRM algorithms. As shown in Section \ref{chap:lr:sec:rrm_80211}, the IEEE 802.11 standard provides a limited set of tools for RRM, leaving assignment algorithms and policies on the behalf of WLAN equipment vendors. As reported in \cite{suiHowBadAre2015}, Cisco's RRM software, shipped with Cisco Aironet APs and Cisco WLAN Controller, was able to improve network performance using Dynamic Channel Selection (DCS) and Transmit Power Control (TPC) so that carrier sense interference was responsible for only 5\% of network delays. RRM solutions from Cisco and other vendors will be surveyed in Section \ref{chap:lr:sec:prop_rrm}. However, since those technologies are proprietary, their implementation details are not disclosed, so they cannot be properly evaluated in independent research or adopted by third-party vendors. Moreover, such solutions lack interoperability, so it is difficult to use them with networking products from other vendors, which is a major obstacle for large-scale WLAN deployments and leads to vendor lock-in situations. Thus, research on RRM algorithms is important for the industry.

\subsection {Radio Resource Management Approaches}
\label{chap:lr:sec:rrm_approaches}
Since RRM is a broad topic that does not imply a single methodology, approach, or even a definition, the classification of RRM algorithms is a challenging task. Most of the papers with the "Radio Resource Management" keyword are focused on problems of cellular networks, such as LTE or 5G. However, in some cases problem formalization, optimization objectives and algorithms can also be useful for research in 802.11 networks. Still, band usage, client management, deployment, and operation specifics make most of the approaches from cellular networks inapplicable for 802.11 networks.
To my knowledge, no comprehensive survey on RRM in 802.11 WLAN exists. I will refer to \cite{bouhafsPerFlowRadioResource2020}, which provides a detailed overview of previous research, and \cite{leeDeepLearningAidedChannel2023}, which describes the state-of-the-art in RRM.
In \cite{bouhafsPerFlowRadioResource2020}, the authors classify the RRM algorithms into three categories:
\begin{itemize}
    \item \textit{per-cell} approaches seek to optimize the RF situation within the AP's cell coverage. This means that adjustments of radio parameters are applied on a cell scale and will be in effect for all stations within the cell. Such a classification can be further divided into the following:
    \begin{itemize}
        \item \textit{localized (uncoordinated) per-cell}, where each AP performs RRM decisions independently;
        \item \textit{centralized per-cell}, where a central entity, such as WLAN Controller, performs RRM decisions for all APs within a WLAN. Some authors refer to this approach as \textit{super-cell} approach \cite{levantiCAPWAPCompliantSolutionRadio2007}; 
        \item \textit{coordinated per-cell}, employing cooperation between APs for making coordinated RRM decisions. 
    \end{itemize} 
    \item \textit{per-link} approaches, which optimize the transmission power for a given station; 
    \item \textit{per-flow} approaches, which employ frequency and AP transmission power adjusting to optimize the QoS to the granularity of a given traffic flow within a station, for example, to the flow of a VoIP application.
\end{itemize}

In fact, a simple localized per-cell RRM is already widely implemented: almost every home Wi-Fi router has the option to select the channel automatically. Typically, in this case, the access point surveys each channel, makes an estimate how congested it is, and then switches to the least congested one. This technique is called \textit{least-congested channel scan}, or \textit{least-congested channel search} (LCCS), and the original design uses the number of associated clients as an estimation of channel congestion \cite{achantaMethodApparatusLeast2006}.
As analyzed in \cite{aruneshmishraWeightedColoringBased2005}, LCCS has several limitations:
\begin{enumerate}
    \item LCCS is unable to accurately identify interference scenarios where clients connected to different Access Points (APs) interfere with each other without the APs themselves causing interference. This issue is particularly prevalent in real-world setups where APs are strategically placed to ensure wide coverage while overlapping minimally to avoid coverage gaps;
    \item LCCS also falls short in optimizing channel reuse based on the distribution of clients. It fails to account for the interference experienced by clients, thus missing the opportunity for channel reuse strategies based on client locations and densities.
\end{enumerate}

In general, uncoordinated decision-making like LCCS tends to yield suboptimal results. Consider an extreme case where a number of APs can sense that channel $C_i$ is not congested and make a decision to switch to that channel. As a result, $C_i$ becomes congested, so APs will seek to switch to another channel $C_j$, where the problem will reoccur. This situation suggests that using a coordinated RRM policy between APs can improve overall WLAN capacity and thus achieve a global optimum.

Research on Transmit Power Control (TPC) methods, which adjust transmission power $P_{Tx}$ to maintain an acceptable Signal-to-Noise-plus-Interference Ratio (SINR), shows potential for enhancing bandwidth. However, concerns regarding these studies' simplified models, unrealistic experimental setups, and statistically uncertain outcomes suggest the need for further investigation \cite{michalskiSimplePerformanceboostingAlgorithm2016,kazminIspolzovanieNeyronnyhSetey2021}. The effect of IEEE 802.11 roaming on TPC is underexplored, especially in scenarios where APs are part of the same extended service set (ESS), which is more common in enterprise WLANs.

As shown in \cite{ramachandranSymphonySynchronousTwophase2008}, per-link TPC considerably improves WLAN performance, achieves more spatial reuse, increases throughput, and able to avoid channel access asymmetry and receiver-side interference (also known as hidden-node problem). However, such an approach has certain hardware requirements, namely \textit{per-packet transmit power control}, a feature available only for a small selection of 802.11 chipsets.
In turn, implementing per-flow RRM in standard 802.11 networks requires an advanced framework to identify specific traffic flows and assess their Quality of Service (QoS) demands \cite{bouhafsPerFlowRadioResource2020}. Therefore, this thesis will focus on solutions that are more practical and applicable to the hardware and software currently available on the market.

\subsection{Mathematical Models for Radio Resource Management}
\label{chap:lr:sec:math_models}

Building on the definition of the Radio Resource Management (RRM) problem introduced in \ref{chap:intro}, we consider a network composed of a set of base stations (access points) denoted as $\boldsymbol{B}$, capable of operating over a collection of channels $\boldsymbol{C}$, each with a maximum transmission power limit $P_{max}$. The core objective of RRM is to establish a radio link between a client device and an access point by assigning a triplet $(b, c, p)$, where $b \in \boldsymbol{B}$ represents the base station, $c \in \boldsymbol{C}$ the channel and $p \leq P_{max}$ the transmission power, so that the network capacity is maximized.

The RRM problem, thus, decomposes into three crucial tasks:
\begin{itemize}
\item \textit{Client Assignment} --- allocating a base station (access point) to a client;
\item \textit{Adaptive Channel Selection} --- determining the optimal frequency (channel) for client communication;
\item \textit{Transmit Power Control} --- setting the appropriate transmission power for client communication.
\end{itemize}

Client Assignment is typically handled by the 802.11 client through roaming decisions, with amendments like 802.11k/r/v designed to enhance and expedite the process of switching to an access point that offers superior service quality. This aspect, therefore, lies beyond the scope of this study.

Channel allocation and transmit power selection, though extensively studied, are often addressed as separate entities in the literature. Combining these factors introduces complexity, as their objectives can conflict. For example, if the objective is to minimize interference, it can be achieved with minimizing transmission power. However, such an approach probably does not satisfy the coverage and quality-of-service requirements. However, considering channel allocation and transmit power control together can also be troublesome, since a change in one variable would change the overall RF situation and the algorithm would not converge.

Another aspect is the choice of metrics. Although the most intuitive and desired metrics are high-level metrics such as network throughput and capacity, actual values of such metrics cannot be used at the time of RRM computations: a significant time of monitoring is required to estimate how throughput changed, so only past records can be used. Another, simpler approach, followed by many works, is to assume that network throughput or capacity is a function of one or more physical layer metrics, such as interference, Received Signal Strength Indicator (RSSI), received signal power, etc. Indeed, a low interference level and low power signal from other APs implies that less transmission errors tend to happen, and more frames can be transmitted with CSMA/CA MAC mechanism.

Reliance on physical layer metrics allows for more immediate adjustments in the RF configuration of a WLAN, ensuring the adaptability of the network to immediate environmental changes.
However, in practice, RRM adjustments can lead to disruptions. Most client devices lack support for the Channel Switch Announcement feature from 802.11h, interpreting a channel switch as if the AP has become unavailable. Therefore, utilizing historical data becomes instrumental in making informed, albeit infrequent, and periodic RRM decisions.

Channel and transmit power settings for each AP should yield optimal values of some given metric for the whole WLAN, such as: interference level, WLAN throughput, WLAN capacity, etc. Thus, most of the works consider RRM as an optimization problem, such as integer linear programming (ILP) \cite{leeOptimizationAPPlacement2002} \cite{rodriguesDesignCapacityPlanning2000} or binary quadratic programming (BQP) \cite{leeDeepLearningAidedChannel2023}.
Since both ILP and BQP are proven to be NP-hard problems, researchers propose heuristics to reduce search space \cite{levantiCAPWAPCompliantSolutionRadio2007}, or apply meta-heuristic methods such as genetic algorithms \cite{raschellaEvaluationChannelAssignment2019} or deep neural networks \cite{leeDeepLearningAidedChannel2023}. Other approaches, while they do not solve the optimization problem explicitly, aim to keep some target metric, such as Signal-to-Interference-plus-Noise Ratio (SINR), within pre-defined acceptable boundaries \cite{michalskiSimplePerformanceboostingAlgorithm2016}.
In \cite{aruneshmishraWeightedColoringBased2005}, authors employ a graph model, where APs are represented as nodes and edges connect APs that can potentially interfere. Using such a model, each node can be assigned a color that represents its channel.

It is important to note that in 802.11 WLANs, all clients connected to a specific Access Point (AP) use the same frequency and transmission power settings. Given the limited applicability of per-link (per-client) Transmit Power Control (TPC), as previously discussed, it is assumed that both frequency and transmission power are configured for the entire cell. This means that all clients of a given AP operate on the same frequency and the AP maintains a consistent transmit power level for communication with all its clients.


\section {Proprietary RRM Solutions}
\label{chap:lr:sec:prop_rrm}
This section reviews proprietary RRM solutions offered by leading vendors in the enterprise WLAN market. I focus on Cisco, Juniper Networks, and Ruckus Networks, since they are the most popular vendors in the enterprise WLAN market \cite{WiFiMarketSize}. To the best of my knowledge, peer-reviewed evaluations of these proprietary RRM efficiency are very limited and scarce, so one could only rely on the claims made by the vendor themselves.

\subsection{Cisco}
Cisco's RRM strategy is integral to its Cisco Centralized Architecture, known as the Unified Wireless Network (UWN) \cite{CiscoUnifiedWirelessa}. In the UWN framework, a single or multiple Wireless LAN Controllers (WLCs) manage up to several thousand Access Points (APs). These WLCs act as the core of the WLAN architecture, enabling centralized control and collection of telemetry from all APs within the network. A WLC can be specialized hardware or a virtual machine hosted in the cloud \cite{arenaUnderstandingTroubleshootingCisco2022}.
Effectively, in UWN, APs can be thought of as Wi-Fi network interface cards for the WLC, providing minimal real-time functionality from the 802.11 standard that cannot be carried out by WLC due to propagation and transmission delays.
This architectural model has become the de facto standard for large-scale enterprise WLANs and is used by most major vendors in the industry.

Cisco offers several RRM solutions. First, CleanAir is a flagship technology from Cisco \cite{CiscoCleanAirTechnology2014} to optimize network performance, avoid jamming, and detect interference sources, including non-802.11 ones. Cisco states that it outperforms competitors through several features:

\begin{itemize}
    \item It utilizes specialized hardware for RF analytics. For example, Cisco Catalyst 9100 Series APs contain a dedicated radio for background RF scanning. This functionality allows for continuous service provision to clients without disrupting the main AP radio transceivers. Additionally, the Cisco RF ASIC, a dedicated chip, enables advanced wireless network analytics and spectrum analysis that are unavailable to conventional Wi-Fi modules;
    \item It classifies and visualizes interference sources thanks to dedicated RF hardware;
    \item Comprehensive WLAN-wide radio resource management, supplying both real-time and historical data at varying levels of granularity;
    \item CleanAir is event-driven, that means it can adapt to changing RF environment and adjust radio parameters in a matter of few minutes, drastically reducing downtime.
\end{itemize}
However, CleanAir is only available for the higher-end models in the Cisco product line, posing limitations for its large-scale deployment. Furthermore, the lack of compatible radio analytics hardware from other vendors and undisclosed implementation details restrict the utility of this technology for integration with non-Cisco equipment.
On the other hand, the Cisco Catalyst product line of WLAN Controllers also provides "regular" RRM functionality that only requires regular Wi-Fi chipset and can be used with all Cisco APs \cite{ciscoRadioResourceManagement}. The trade-off for this convenience is access to less detailed information about the RF environment and the necessity for APs to temporarily switch off their current channel to conduct scanning. In this case, APs collect statistics on their current channel whenever they are not transmitting data. In addition, APs periodically scan other channels to collect statistics \cite{arenaUnderstandingTroubleshootingCisco2022}. During scanning, the AP becomes unavailable to its clients, resulting in increased latency for them.

Cisco RRM employs a super-cell concept. In such a scheme, a group of geographically close APs (forming an \textit{RF Group}) is managed by a designated WLAN Controller (\textit{RF Group Leader}).
A subgroup unit within an RF group is called \textit{RF Neighborhood}, and consists of AP that can hear each other at signal strength $\geq -80 \; dBm$ \cite{arenaUnderstandingTroubleshootingCisco2022}. Each AP is associated with two lists: RX neighbors, that is, the list of APs that a given AP can hear, and TX neighbors, a list of APs that can hear the given AP. Cisco RRM maintains a \textit{ cost metric} for every channel, which is an assessment of the channel's quality derived from RSSI, co-channel interference, and non-WiFi interference.


\subsection{Juniper Networks}
Juniper Networks offers Mist AI RRM technology to improve network performance. The notable features are \cite{junipernetworksUnderstandingRadioResource2023,RadioManagementTechnology}:
\begin{itemize}
    \item Automatic dual-band radio management --- if RRM system finds 2.4-GHz radio transmitter to be unused on a given AP, it disables the radio to free airspace for other access points;
    \item Juniper Mist APs incorporate the Predictive Analytics and Correlation Engine (PACE) "to monitor conditions and make out-of-band adjustments" \cite{RadioManagementTechnology};
    \item Telemetry is sent to the Juniper Mist Cloud, so that the cloud can periodically fine-tune APs based on historical data and usage statistics;
    \item Employing a Reinforcement Learning (RL) methodology for the strategic planning of channel selection and power settings across APs in a WLAN, aiming for optimal network performance \cite{junipernetworksUnderstandingRadioResource2023}.
\end{itemize}

\subsection{Ruckus Networks}
Ruckus Networks offers ChannelFly RRM technology, which provides automatic channel selection. ChannelFly estimates the capacity of each channel by continuously monitoring the activity of each channel across the 2.4 and 5 GHz bands. Based on this information, ChannelFly develops a statistical model to predict which channel will offer the highest capacity for clients, as detailed in \cite{RuckusChannelFlyFeature2023}. A key benefit of ChannelFly is its ability to avoid "dead time", defined as the period an AP spends scanning different channels during which it cannot communicate with clients. This capability implies the inclusion of dedicated scanning radios in Ruckus APs, allowing continuous communication with clients while performing channel assessments.

Additionally, Ruckus offers "smart adaptive antenna array" technology. This feature enhances the beamforming of signals from Ruckus APs, focusing the transmission towards clients to improve the Signal-to-Noise Ratio (SNR).

\subsection{Aruba Networks}
The Adaptive Radio Management (ARM) technology by Aruba represents an earlier approach to Radio Resource Management (RRM), utilizing Adaptive Channel Selection and Transmit Power Control to enhance the RF environment in WLANs. ARM stands out for its algorithmic simplicity and the thoroughness of its documentation provided by Aruba, in contrast to other vendors \cite{ArubaOSUserGuide}.

The key features of the ARM include \cite{ARMOverview}:
\begin{itemize}
\item \textbf{Application Awareness}: Addressing the "dead time" caused by APs during channel scanning, ARM throttles the frequency of background scans based on current traffic load, reducing scans under heavy traffic and resuming normal scanning rates when traffic diminishes \cite{UnderstandingARM}.
\item \textbf{Mode Awareness}: To mitigate interference in environments with densely installed APs, ARM can switch excessive APs to Air Monitor mode, where they continuously collect and send RRM telemetry to the controller.
\item \textbf{Band Steering}: Promotes the use 5 GHz band to clients, instead of more congested and higher-range 2.4 GHz band.
\item \textbf{802.11n HT Mode Support}: ARM can utilize a 40 MHz channel pair for 802.11n networks, selecting the best primary and secondary operating channels automatically.
\item \textbf{Noise and Error Monitoring}: Distinguishes between 802.11 and non-802.11 noise sources, improving network reliability.
\item \textbf{Spectrum Load Balancing}: Analyzes client distribution across neighboring APs to direct new connections to less burdened APs, though clients may reconnect to their original choice upon a subsequent attempt.
\item \textbf{Noise Interference Immunity}: Adjusts the receiver sensitivity threshold to ignore weak and non-802.11 signals, reducing unnecessary decoding efforts and improving network performance.
\end{itemize}

However, reports from system administrators suggest that RRM decisions in Aruba ARM are made by APs rather than by the controller. Some other user complains include unnecessary disabling of 2.4GHz radios and erroneous TPC that leads to coverage holes \cite{TamingArubaARM2012}.

However, ARM is a legacy technology. Its successor, Aruba AirMatch, introduced in recent ArubaOS versions, is a more sophisticated RRM technology, which is based on AI and machine learning and is able to perform channel and power planning on a WLAN-wide scale, suggesting ARM was implemented in a per-cell way and AirMatch is a super-cell solution.

Notable AirMatch features:

\begin{itemize}
    \item Channel width adjustment based on device density - the more devices are connected to an AP, the narrower channel width is used to allow channel reuse and reduce interference;
    \item APs measure RF environment for 5 minutes every 30 minutes;
    \item Decisions based on a 24-hour period analytics unlike instant RF situation snapshots in ARM;
    \item Elimination of coverage holes based on TPC.
    \item Configurable thresholds in channel quality improvements to trigger channel and EIRP planning, default threshold is 15\%.
    \item ClientMatch technology that manages clients: performs load balancing between APs, encourages clients to switch to APs providing better signal strength and using higher bands (5 GHz or event 6 GHz in 802.11ax)
\end{itemize}

Similarly to ARM, Aruba provides more information about AirMatch operating logic than other vendors about their RRM solutions.

According to \cite{ArubaOSUserGuide}, AirMatch blacklists the channel for channel selection if a radar was detected on it (in the 5 GHz case) or in the case if a high noise level was detected on it (for all bands). In those cases, AirMatch will select the channel with a minimum interference index.

It is not clear whether AirMatch uses the same metrics as ARM, but only ARM metrics are described in the documentation.
To make RRM decisions, ARM uses two metrics:
\begin{itemize}
    \item \textbf{Coverage Index} - comprises two components, $x$ and $y$:
    \begin{itemize}
        \item $x$ is weighted calculation of Signal-to-Noise Ratio (SNR) for all APs on a given channel;
        \item $y$ is the weighted sum of the SNRs that neighboring APs within the group observe on the same channel.
    \end{itemize}
    \item \textbf{Interference Index} - measures co-channel and adjacent-channel interference, calculated as a sum of four quantities $a$, $b$, $c$, $d$:
    \begin{itemize}
        \item $c$ is the channel interference the AP neighbors see on the selected channel;
        \item $d$ is the interference the AP neighbors see on the adjacent channel.
    \end{itemize}
\end{itemize}
Furthermore, Aruba APs collect several other metrics, including L2 metrics \cite{ArubaOSUserGuide}:

\begin{itemize}
    \item Number of Retry frames, measured in \%
    \item Number of Low-speed frames, measured in \%
    \item Number of Non-unicast frames, measured in \%
    \item Number of Fragmented frames, measured in \%
    \item Number of Bandwidth seen on the channel, measured in kbps
    \item Number of PHY errors seen on the channel measured in \%
    \item Number of MAC errors seen on the channel measured in \%
    \item Value of noise floor on the specified AP
\end{itemize}

Aruba documentation indicates that these metrics offer a "snapshot of the current RF health state" \cite{ARMMetrics}, suggesting that they are informational tools for network administrators rather than being actively used in RRM decision-making.

% \section {Concerns and Requirements for an RRM algorithm}

\section {Conclusion}
\label{chap:lr:sec:conclusion}
Summarizing the insights from previous sections, I can conclude that the problem of radio resource management in 802.11 WLANs is still relevant, since the IEEE 802.11 standard provides only limited tools for RRM, while existing commercial solutions are proprietary and lack interoperability. Thus, there is a need for a novel RRM algorithm that can be implemented in existing enterprise WLAN infrastructure and improve overall network performance.

I find super-cell approach most fitting for a modern RRM algorithm that can be applied in real-world WLAN deployments. Super-cell algorithms, while being practical and having less obstacles in hardware and current device drivers compared to other approaches, still have the potential to vastly improve the RF situation and, thus, WLAN performance.

Centralized management that is typically utilized in super-cell RRM is the standard approach when building modern WLANs, allowing to gather more information about RF environment and come up with more optimal allocations compared with local RRM decision-making.
Moreover, in the long term, the presence of WLC, as a centralized entity with orders of magnitude higher computation power and ability to collect and store statistics from all APs throughout the WLAN, releases the burden of RRM from APs and potentially improves the overall network efficiency.

Despite the promising capabilities of per-link and per-flow radio resource management approaches for optimizing wireless networks in a more fine-grained and application-aware manner, they currently have considerable limitations that prevent them from being implemented in production wireless networking solutions.

As a summary of this survey, we can identify the research gap: the problem of radio resource management in 802.11 WLANs is still relevant, since IEEE 802.11 standard does not provide fully-fledged RRM, while existing commercial solutions are proprietary and lack interoperability. Thus, there is a need for an RRM algorithm addressing key issues, including:
\begin{itemize}
    \item Design for centralized management of enterprise WLAN, working as a part of Wireless LAN Controller;
    \item Applicability with current hardware and software, namely:
    \begin{itemize}
        \item Effortless integration with OpenWRT-based APs;
        \item Requires data like physical and link-layer statistics that can be obtained using only regular Linux Wi-Fi drivers like \texttt{nl80211}, and standard Linux networking tools;
    \end{itemize}
    \item Performing not worse than existing RRM algorithm \textit{RRMGreedy}, analyzed in Chapter \ref{sec:baseline};
    \item Able to combine both channel selection and transmit power adjustment to improve RF environment and network performance.
\end{itemize}

The following chapters will focus on analyzing the limitations of current algorithms and developing a new one.