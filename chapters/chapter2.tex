\chapter{Literature Review}
\label{chap:lr}
\chaptermark{}
The purpose of this chapter is to explore existing approaches on Radio Resource Management (RRM) in IEEE 802.11 networks, including surveying what is proposed as a part of the 802.11 standard itself, what research has been done and what is offered by existing commercial solutions. This chapter is organized as follows:
\begin{itemize}
    \item Section \ref{chap:lr:sec:rrm_80211} provides an overview how radio resource management is facilitated within the IEEE 802.11 standard;
    \item Section \ref{chap:lr:sec:prev_works} provides a synthesis on previous research in radio resource management;
    \item Section \ref{chap:lr:sec:prop_rrm} provides an overview of proprietary RRM solutions from major vendors;
    \item Section \ref{chap:lr:sec:conclusion} summarizes the chapter.
\end{itemize}
Table \ref{defs_table} contains the list of definitions used in this chapter.

\section {Radio Resource Management in IEEE 802.11}
\label{chap:lr:sec:rrm_80211}
As discussed in Section \ref{chap:intro:sec:rrm}, a radio resource allocation algorithm aims to optimize network capacity through optimizing spectrum usage by adjusting two parameters: frequency and transmission power of each access point. The very need for an RRM algorithm comes from situation called \textit{Overlapping Basic Service Sets} in the IEEE 802.11 standard, when there are two basic service sets operating on the same channel within reach of each other. This situation leads to co-channel interference, which degrades network performance. In this section, I provide an overview of methods and techniques provided by IEEE 802.11 standard that can assist in radio resource management.


\subsection {IEEE 802.11h}
\label{chap:lr:sec:80211h}
To comply with legal requirements on transmissions in 5 GHz, 802.11h-2003 \cite{80211h} amendment (\textit{Spectrum and Transmit Power Management Extensions}) was introduced. Since the goal of this amendment is to prevent legacy 802.11a 5GHz APs from interference with radars, 802.11h is not oriented for optimizing capacity of a wireless network. However, 802.11h introduces \cite{konsgenSpectrumManagementAlgorithms2010} spectrum management methods, namely, \textit{Dynamic Frequency Selection (DFS)}, facilitating automatic change of AP's operating frequency, and \textit{Transmit Power Control (TPC)}, adjusting the power of AP's transmitter.
Those methods, however, do not implement RRM in the sense of optimizing network capacity, but rather are aimed at preventing interference with radars operating on the 5 GHz band. 802.11h describes procedures for: quieting the channel to detect presence of a radar, switching to another channel if a radar detecting, advertising the channel switch to client stations (STAs). Radar detection itself, though, is beyond the scope of 802.11 standard \cite{80211h}.
However, the methods introduced in 802.11h can be considered as a foundation for further research on RRM algorithms.


% \subsubsection{Transmit Power Control}

\subsection {IEEE 802.11k}
\label{chap:lr:sec:80211k}
The amendment 802.11k-2008 \cite{80211k} (\textit{Radio Resource Measurement}, not to be confused with \textit{Radio Resource Management}) improves the performance of roaming. Roaming is a process of station moving from one access point to another \cite{colemanCWNACertifiedWireless2021}. The amendment 802.11k improves roaming by allowing stations to request from access points various reports, such as \textit{neighbor reports} for discovering possible access points that STA can roam to, \textit{link measurements} to estimate how well AP can hear the station etc. This allows stations to reduce power consumption, speed up roaming and decrease power consumption and airtime usage spent on sending probe requests to each channel when trying to find another AP to roam.
However, this amendment is also not aimed at optimizing network capacity, but rather at improving roaming performance.

\subsection {IEEE 802.11ax}
\label{chap:lr:sec:80211ax}

The amendment 802.11ax \cite{80211ax} (\textit{High Efficiency WLAN}) introduces means to address OBSS problem. The amendment introduces \textit{Basic Service Set (BSS) Coloring}, a method to distinguish between different BSSs operating on the same channel. It works as following: each BSS is assigned with \textit{color}, a 6-bit value spanning from 1 to 63, that resides in the PHY header. Devices belonging to the same BSS check this color when demodulating transmissions (i.e., check if this frame is an \textit{intra-BSS}). If it contains the wrong color (\textit{inter-BSS} frame), further demodulation is not performed, thus saving processing time \cite{CiscoCatalyst9800}.
If an intra-BSS frame is found to use the same color, AP switches to another color.
One can see that such solution completely relies on CSMA/CA media access control (MAC) mechanism and is not able to address hidden terminal problem. That is, if AP providing overlapping BSS is a hidden terminal, the signal will be disrupted near the destination, so the receiving station will not be able to demodulate the signal to inspect the color tag.
Thus, this technique does not solve the problem of co-channel interference and does not provide channel planning, so this cannot be considered as a complete RRM solution.



\section {Previous Works}
\label{chap:lr:sec:prev_works}
Interference from other Access Points poses a serious obstacle \cite{suiHowBadAre2015} for delivering acceptable quality of service in large 802.11 WLAN deployments. Although the 802.11 carrier-sense MAC protocol is designed to be resilient to interference, interference reduces the available airtime and causes loss of frames already sent. Thus, network capacity and performance tend to degrade drastically \cite{levantiCAPWAPCompliantSolutionRadio2007}. For interference between cells within a single WLAN, such problem in principle can be solved by proper \cite{site surveying}, i.e., planning of geographical placement of cells. However, another and the major source of interference are \text{rogue access points} (RAPs), which are operated by third parties and in general are not under control of WLAN administrators. According to \cite{suiHowBadAre2015}, interference from rogue APs can introduce up to 50\% delays in a WLAN. Moreover, the prevalent amount (more than 70\%) of rogue APs are stationary \cite{suiHowBadAre2015}, so their radio presence can be considered as a constant factor in the WLAN. Note that \textit{rogue} does not imply that those APs are malicious or posing threats other than congesting channels and occuping airtime as a result of their legitimate operation.
In this light, attempts to improve spectrum management via channel assignment and transmit power control adjustment algorithms encompass research on Radio Resource Management algorithms. As shown in Section \ref{chap:lr:sec:rrm_80211}, the IEEE 802.11 standard provides a limited set of tools for radio resource managemens, leaving assignment algorithms and policies on behalf of WLAN equipment vendors. As reported in \cite{suiHowBadAre2015}, Cisco's RRM software, shipped with Cisco Aironet APs and Cisco WLAN Controller, was able to improve network performance using Dynamic Channel Selection (DFS) and Transmit Power Control (TPC)  so that carrier sense interference was responsible for only 5\% of network delays. RRM solutions from Cisco and other vendors will be surveyed in Section \ref{chap:lr:sec:prop_rrm}. However, since those technologies are proprietary, their implementation details are not disclosed, so they cannot be properly evaluated in an independent research or adopted by third-party vendors. Moreover, such solutions lack interoperability, so it is difficult to use them with networking products from other vendors, which is a major obstacle for large-scale WLAN deployments and leads to vendor lock-in situations. Thus, research on radio resource management algorithms is important for the industry.

\subsection {Radio Resource Management Approaches}
\label{chap:lr:sec:rrm_approaches}
As RRM is a broad topic which does not imply a single methodology or approach, classification of RRM algorithms is a challenging task. Most of the papers with the "Radio Resource Management" keyword are focused on problems of cellular networks, such as LTE or 5G. At the same time, the problem formalization, some optimization objective and algorithms can also be used for research in 802.11 networks, but band usage, client management, deployment and operation specifics make most of the proposals inapplicable for 802.11 networks.
To the best of my knowledge, no comprehensive survey on RRM in 802.11 WLAN exist. I will refer to \cite{bouhafsPerFlowRadioResource2020}, which provides detailed overview of previous research, and \cite{leeDeepLearningAidedChannel2023}, describing state-of-the-art on RRM.
In \cite{bouhafsPerFlowRadioResource2020}, authors classify RRM algorithms into three categories:
\begin{itemize}
    \item \textit{per-cell} approaches seek to optimize the RF situation within the AP's cell coverage. This means that adjustments of radio parameters applied on a cell scale and will be in effect for all stations within the cell. Such classification can be further divided into:
    \begin{itemize}
        \item \textit{localized (uncoordinated) per-cell}, where each AP performs RRM decisions independently;
        \item \textit{centralized per-cell}, where a central entity, such as WLAN Controller, performs RRM decisions for all APs within a WLAN. Some authors refer to this approach as \textit{super-cell} approach \cite{levantiCAPWAPCompliantSolutionRadio2007}; \item \textit{coordinated per-cell}, employing cooperation between APs for making coordinated RRM decisions.  \end{itemize} \item \textit{per-link} approaches, which optimize the transmission power for a given station; \item \textit{per-flow} approaches, which employ frequency and AP Tx power adjusting to optimize the QoS to the granularity of a given traffic flow within a station, for example, to the flow of a VoIP application.
\end{itemize}

In fact, a simple localized per-cell RRM is already widely implemented: almost every home Wi-Fi router has the option to select channel automatically. Typically, in this case access point surveys each channel, makes an estimation how congested it is, then switches to the least congested one. This technique is called \textit{least-congested channel scan}, or \textit{least-congested channel search} (LCCS), and the original design uses the number of associated clients as estimation of channel congestion \cite{achantaMethodApparatusLeast2006}.
As analyzed in \cite{aruneshmishraWeightedColoringBased2005}, LCCS has several limitations:
\begin{enumerate}
    \item LCCS is unable to accurately identify interference scenarios where clients connected to different Access Points (APs) interfere with each other without the APs themselves causing interference. This issue is particularly prevalent in real-world setups where APs are strategically placed to ensure wide coverage while overlapping minimally to avoid coverage gaps;
    \item LCCS also falls short in optimizing channel reuse based on the distribution of clients. It fails to account for the interference experienced by clients, thus missing the opportunity for channel reuse strategies based on client locations and densities.
\end{enumerate}

In general, uncoordinated decision-making like LCCS tends to yield suboptimal results. Consider an extreme case, where a number of APs can sense that channel $C_i$ is not congested and make a decision to switch to that channel. As a result, $C_i$ becomes congested, so APs will seek to switch to another channel $C_j$, where the problem will reoccur. This situation displays that RRM algorithms need a certain degree of coordination between APs using an algorithm aiming to improve overall WLAN capacity and thus achieve a global optimum.

Research on Transmit Power Control (TPC) methods, which adjust transmission power $P_{Tx}$ to maintain an acceptable Signal-to-Noise-plus-Interference Ratio (SINR), shows potential for enhancing bandwidth. However, criticisms regarding these studies' simplified models, unrealistic experimental setups, and statistically uncertain outcomes suggest the need for further investigation \cite{michalskiSimplePerformanceboostingAlgorithm2016,kazminIspolzovanieNeyronnyhSetey2021}. The effect of IEEE 802.11 roaming on TPC is underexplored, especially in scenarios where APs are part of the same extended service set (ESS), which is more common in enterprise WLANs.

As shown in \cite{ramachandranSymphonySynchronousTwophase2008}, per-link TPC considerably improves WLAN performance, achieves more spatial reuse, increases throughput, and able to avoid channel access asymmetry and receiver-side interference (also known as hidden-node problem). However, such approach has certain hardware requirements, namely, \textit{per-packet transmit power control}, a feature available only for a small selection of 802.11 chipsets.
In turn, implementing per-flow RRM in standard 802.11 networks requires an advanced framework for identifying specific traffic flows and assessing their Quality of Service (QoS) demands \cite{bouhafsPerFlowRadioResource2020}. Therefore, this thesis will concentrate on solutions that are more practical and applicable to the hardware and software currently available on the market.

\subsection{Mathematical Models for Radio Resource Management}
\label{chap:lr:sec:math_models}

Building on the definition of the Radio Resource Management (RRM) problem introduced in \ref{chap:intro}, we consider a network composed of a set of base stations (access points) denoted as $\boldsymbol{B}$, capable of operating over a collection of channels $\boldsymbol{C}$, each with a maximum transmission power limit $P_{max}$. The core objective of RRM is to establish a radio link between a client device and an access point by assigning a triplet $(b, c, p)$, where $b \in \boldsymbol{B}$ signifies the base station, $c \in \boldsymbol{C}$ the channel, and $p \leq P_{max}$ the transmission power, so that network capacity is maximized.

The RRM problem, thus, decomposes into three crucial tasks:
\begin{itemize}
\item \textit{Client Assignment} --- allocating a base station (access point) to a client;
\item \textit{Adaptive Channel Selection} --- determining the optimal frequency (channel) for client communication;
\item \textit{Transmit Power Control} --- setting the appropriate transmission power for client communication.
\end{itemize}

Client Assignment is typically handled by the 802.11 client through roaming decisions, with amendments like 802.11k/r/v designed to enhance and expedite the process of switching to an access point that offers superior service quality. This aspect, therefore, lies beyond the scope of this study.

Channel allocation and transmit power selection, though extensively studied, are often addressed as separate entities in the literature. Combining these factors introduces complexity, as their objectives can conflict. For example, if the objective is to minimize interference, it can be achieved with minimizing the transmission power. However, such outcome probably does not satisfy coverage and quality of service requirements. On the other hand, considering channel and transmit power together can also be troublesome, since change in one variable would change the overall RF situation, and the algorithm would not converge.

Another aspect is metric selection. Although the most intuitive and desired metrics are high-level ones like network throughput and capacity, actual values of such metrics cannot be used at the time of RRM computations: a significant time of monitoring is required to estimate how throughput changed, so only past records can be used. Another, simpler approach, followed by many works, is to assume that network throughput or capacity is a function of one or more physical layer metrics, such as interference, Received Signal Strength Indicator (RSSI), Rx signal power etc. Indeed, low interference level and low power signal from other APs implies that less transmission errors tend to happen, and more frames can be transmitted with CSMA/CA MAC mechanism.
This reliance on physical layer metrics allows for more immediate adjustments in RF (Radio Frequency) configuration of a WLAN, ensuring the adaptability of the network to immediate environmental changes.
However, RRM adjustments in practice can lead to disruptions. Most client devices lack support for the Channel Switch Announcement feature from 802.11h, interpreting a channel switch as if the AP has become unavailable. Therefore, utilizing historical data becomes instrumental in making informed, albeit infrequent and periodic, RRM decisions.

Thus, channel and transmit power settings for each AP should yield optimal value of some given metric for the whole WLAN, such as: interference level, WLAN throughput, WLAN capacity, etc. Thus, most of the works consider RRM as an optimization problem, such as integer linear programming (ILP) \cite{leeOptimizationAPPlacement2002} \cite{rodriguesDesignCapacityPlanning2000} or binary quadratic programming (BQP) \cite{leeDeepLearningAidedChannel2023}.
Since both ILP and BQP are proven to be NP-hard problems, researchers propose heuristics to reduce search space \cite{levantiCAPWAPCompliantSolutionRadio2007}, or apply meta-heuristic methods such as genetic algorithms \cite{raschellaEvaluationChannelAssignment2019} or deep neural networks \cite{leeDeepLearningAidedChannel2023}. Other approaches, while not solving optimization problem explicitly, aim to keep some target metric, such as SINR (Signal-to-Interference-plus-Noise Ratio), within pre-defined acceptable boundaries \cite{michalskiSimplePerformanceboostingAlgorithm2016}.
In \cite{aruneshmishraWeightedColoringBased2005}, authors employ a graph model, where APs are represented as nodes and edges connect APs which can potentially interfere. Using such model, each node can be assigned a color representing its channel.

It is important to emphasize that in 802.11 WLANs, all clients connected to a specific Access Point (AP) utilize the same frequency and transmission power settings. Given the limited applicability of per-link (per-client) Transmit Power Control (TPC), as previously discussed, it is assumed that both frequency and transmission power are configured for the entire cell. This means that all clients of a given AP operate on the same frequency, and the AP maintains a consistent transmit power level for communication with all its clients.


\section {Proprietary RRM Solutions}
\label{chap:lr:sec:prop_rrm}
This section surveys proprietary RRM solutions offered by leading vendors in the enterprise WLAN market. I focus on Cisco, Juniper Networks, and Ruckus Networks, since they are the most popular vendors in enterprise WLAN market \cite{WiFiMarketSize}. To the best of my knowledge, peer-reviewed evaluations of those proprietary RRM efficiency are very limited and scarce, so one could only rely on the claims made by vendor themselves.

\subsection{Cisco}
Cisco's RRM strategy is integral to its Cisco Centralized Architecture, known as the Unified Wireless Network (UWN) \cite{CiscoUnifiedWirelessa}. In the UWN framework, a single or multiple Wireless LAN Controllers (WLCs) manage up to several thousand Access Points (APs). These WLCs act as the core of the WLAN architecture, enabling centralized control and the collection of telemetry from all APs within the network. A WLC can be either specialized hardware or a virtual machine hosted in the cloud \cite{arenaUnderstandingTroubleshootingCisco2022}.
Effectively, in UWN, access points can be thought of as Wi-Fi network interface cards for the WLC, providing minimal real-time functionality from 802.11 standard that cannot be carried out to WLC due to propagation and transmission delays.
This architectural model has become the de facto standard for large-scale enterprise WLANs and is employed by most major vendors in the industry.

Cisco offers several RRM solutions. First, CleanAir is a flagship technology from Cisco \cite{CiscoCleanAirTechnology2014} to optimize network performance, avoid jamming, and detect interference sources, including non-802.11 ones. Cisco states that it outperforms competitors through several features:

\begin{itemize}
    \item It utilizes specialized hardware for RF analytics. For instance, the Cisco Catalyst 9100 Series Access Points contain a scanning radio for background RF scanning. This functionality allows for continuous service provision to clients without disrupting the main AP radio transceivers. Additionally, the Cisco RF ASIC, a dedicated chip, enables advanced wireless network analytics and spectrum analysis unavailable to conventional Wi-Fi modules;
    \item Classifying and visualizing interference sources thanks to dedicated RF hardware;
    \item Comprehensive WLAN-wide radio resource management, supplying both real-time and historical data at varying levels of granularity;
    \item CleanAir is event-driven, that means it can adapt to changing RF environment and adjust radio parameters in a matter of few minutes, drastically reducing downtime.
\end{itemize}
However, CleanAir is only available for the higher-end models in the Cisco product line, posing limitations for its large-scale deployment. Furthermore, lack of compatible radio analytics hardware from other vendors and undisclosed implementation details restrict the utility of this technology for integration with non-Cisco equipment.
On the other hand, Cisco Catalyst product line of WLAN Controllers also provide "regular" RRM functionality that only requires regular Wi-Fi chipset and can be used with all Cisco APs \cite{ciscoRadioResourceManagement}. The trade-off for this convenience is access to less detailed information about the RF environment and the necessity for Access Points to temporarily switch off their current channel to conduct scanning. In this case, APs collect statistics on their current channel any time they are not transmitting data. Additionally, periodically APs scan other channels to gather statistics. \cite{arenaUnderstandingTroubleshootingCisco2022}. At that time, AP is not available to clients, so, scanning introduces latencies for the clients connected to the AP.

Cisco RRM employs a super-cell concept. In such scheme, a group of geographically close APs (forming an \textit{RF Group}) is managed by a designated WLAN Controller (\textit{RF Group Leader}).
A subgroup unit within an RF group is called \textit{RF Neighborhood}, and consists of AP that can hear each other at signal strength $\geq 80 \; dBm$ \cite{arenaUnderstandingTroubleshootingCisco2022}. Each AP is associated with two lists: RX neighbors, i.e., list of APs that given AP can hear, and TX neighbors, a list of APs that can hear given AP.
For each channel, Cisco RRM maintains a \textit{cost metric}, an estimation of channel goodness that is based on RSSI, co-channel interference, non-WiFi interference.


\subsection{Juniper Networks}
Juniper Networks offers Mist AI RRM technology to improve network performance. The notable features are \cite{junipernetworksUnderstandingRadioResource2023,RadioManagementTechnology}:
\begin{itemize}
    \item Automatic dual-band radio management --- if RRM system finds 2.4-GHz radio transmitter to be unused on a given AP, it disables the radio to free airspace for other access points;
    \item Juniper Mist APs incorporate the Predictive Analytics and Correlation Engine (PACE) "to monitor conditions and make out-of-band adjustments" \cite{RadioManagementTechnology};
    \item Telemetry is sent to the Juniper Mist Cloud, so that the cloud can periodically fine-tune APs based on historical data and usage statistics;
    \item Employing a Reinforcement Learning (RL) methodology for the strategic planning of channel selection and power settings across APs in a WLAN, aiming for optimal network performance \cite{junipernetworksUnderstandingRadioResource2023}.
\end{itemize}

\subsection{Ruckus Networks}
Ruckus Networks offers ChannelFly RRM technology, which provides automatic channel selection. ChannelFly estimates capacity of each channel by continuously monitoring the activity of each channel across the 2.4 and 5GHz bands. Based on this information, ChannelFly develops a statistical model to predict which channel will offer the highest capacity for clients, as detailed by \cite{RuckusChannelFlyFeature2023}. A key benefit of ChannelFly is its ability to avoid "dead time", defined as the period an AP spends scanning different channels, during which it cannot communicate with clients. This capability implies the inclusion of dedicated scanning radios in Ruckus APs, allowing continuous communication with clients while performing channel assessments.

Additionally, Ruckus offers a "smart adaptive antenna array" technology. This feature enhances the directionality of signals from Ruckus APs, focusing the transmission towards clients to improve the Signal-to-Noise Ratio (SNR).

\subsection{Aruba Networks}
The Adaptive Radio Management (ARM) technology by Aruba represents an earlier approach to Radio Resource Management (RRM), utilizing Adaptive Channel Selection and Transmit Power Control to enhance the RF environment in WLANs. ARM stands out for its algorithmic simplicity and the thoroughness of its documentation provided by Aruba, in contrast with other vendors \cite{ArubaOSUserGuide}.

Key features of ARM include \cite{ARMOverview}:
\begin{itemize}
\item \textbf{Application Awareness}: Addressing the "dead time" caused by APs during channel scanning, ARM throttles the frequency of background scans based on current traffic load, reducing scans under heavy traffic and resuming normal scanning rates when traffic diminishes \cite{UnderstandingARM}.
\item \textbf{Mode Awareness}: To mitigate interference in environments with densely installed APs, ARM can switch excessive APs to Air Monitor mode, where they continuously collect and send RRM telemetry to the controller.
\item \textbf{Band Steering}: Encourages dual-band capable clients to prefer the 5GHz band to alleviate congestion on the 2.4GHz band.
\item \textbf{802.11n HT Mode Support}: ARM can utilize a 40 MHz channel pair for 802.11n networks, selecting the best primary and secondary operating channels automatically.
\item \textbf{Noise and Error Monitoring}: Distinguishes between 802.11 and non-802.11 noise sources, improving network reliability.
\item \textbf{Spectrum Load Balancing}: Analyzes client distribution across neighboring APs to direct new connections to less burdened APs, though clients may reconnect to their original choice upon a subsequent attempt.
\item \textbf{Noise Interference Immunity}: Adjusts the receiver sensitivity threshold to ignore weak and non-802.11 signals, reducing unnecessary decoding efforts and improving network performance.
\end{itemize}

Reports from system administrators, though, suggest that RRM decisions in Aruba ARM are made by APs rather than controller. Among other user complains are unnecessary disabling of 2.4GHz radios, erroneous TPC leading to coverage holes \cite{TamingArubaARM2012}.

However, ARM is a legacy technology. Its successor, Aruba AirMatch, introduced in recent ArubaOS versions, is a more sophisticated RRM technology, which is based on AI and machine learning and is able to perform channel and power planning on a WLAN-wide scale, suggesting ARM was implemented in a per-cell way and AirMatch is a super-cell solution.

Notable AirMatch features:

\begin{itemize}
    \item Channel width adjustment based on device density - the more devices are connected to an AP, the narrower channel width is used to allow channel reuse and reduce interference;
    \item APs measure RF environment for 5 minutes every 30 minutes;
    \item Decisions based on a 24-hour period analytics unlike instant RF situation snapshots in ARM;
    \item Elimination of coverage holes based on TPC.
    \item Configurable thresholds in channel quality improvements to trigger channel and EIRP planning, default threshold is 15\%.
    \item ClientMatch technology that manages clients: performs load balancing between APs, encourages clients to switch to APs providing better signal strength and using higher bands (5 GHz or event 6 GHz in 802.11ax)
\end{itemize}

Similarly to ARM, Aruba provides more information about AirMatch operating logic than other vendors about their RRM solutions.

According to \cite{ArubaOSUserGuide}, AirMatch blacklists channel for channel selection if a radar was detected on it (in 5 GHz case) or in case if high noise level was detected on it (for all bands). In those cases, AirMatch will select channel with a manimum interference index.

It is not clear if AirMatch uses the same metrics as ARM, but only ARM metrics are described in the documentation.
To make RRM decisions, ARM uses two metrics:
\begin{itemize}
    \item \textbf{Coverage Index} - calculated a $\frac{x}{y}$, where $x$ is the AP's weighted calculation of SNR on all valid APs on a specified 802.11 channel, and $y$ is the weighted calculation of the AP's SNR the neighboring APs see on that channel.
    \item \textbf{Interference Index} - metric to measure co-channel and adjacent-channel interference, calculated as a sum of four quantities $a$, $b$, $c$, $d$:
    \begin{itemize}
        \item $c$ is the channel interference the AP neighbors see on the selected channel.
        \item $d$ is the interference the AP neighbors see on the adjacent channel.
    \end{itemize}
\end{itemize}
Additionally, Aruba APs collect several other metrics, including L2 metrics:

\begin{itemize}
    \item Amount of Retry frames (measured in \%)
    \item Amount of Low-speed frames (measured in \%)
    \item Amount of Non-unicast frames (measured in \%)
    \item Amount of Fragmented frames (measured in \%)
    \item Amount of Bandwidth seen on the channel (measured in kbps)
    \item Amount of PHY errors seen on the channel (measured in \%)
    \item Amount of MAC errors seen on the channel (measured in \%)
    \item Noise floor value for the specified AP
\end{itemize}

Aruba documentation indicates that these metrics offer a "snapshot of the current RF health state" \cite{ARMMetrics}, suggesting they are informational tools for network administrators rather than being actively used in RRM decision-making.

\section {Conclusion}
\label{chap:lr:sec:conclusion}
Summarizing the insights from prior sections, I can conclude that the problem of radio resource management in 802.11 WLANs is still relevant, since the IEEE 802.11 standard provides only limited tools for RRM, while existing commercial solutions are proprietary and lack interoperability. Thus, there is a need for a novel RRM algorithm that can be implemented in existing enterprise WLAN infrastructure and improve overall network performance.

I find super-cell approach most fitting for a modern RRM algorithm that can be applied in real-world WLAN deployments. Super-cell algorithms, while being practical and having less obstacles in hardware and current device drivers compared to other approaches, still have the potential to vastly improve RF situation and, thus, WLAN performance.

Centralized management that is typically utilized in super-cell RRM is the standard approach when building modern WLANs, allowing to gather more information about RF environment and come up with more optimal allocations compared with local RRM decision-making.
Moreover, presence of WLC as centralized entity with orders of magnitude higher computation power and ability to collect and store statistics from all APs all over the WLAN in the long term releases the burden of RRM from Access Points and potentially improves the overall network efficiency.

Despite the promising capabilities of per-link and per-flow radio resource management approaches for optimizing wireless networks in a more fine-grained and application-aware manner, they have considerable limitations that currently prevent from implementing them in production wireless networking solutions.

As a summary of this survey, we can identify the research gap: the problem of radio resource management in 802.11 WLANs is still relevant, since IEEE 802.11 standard does not provide fully-fledged RRM, while existing commercial solutions are proprietary and lack interoperability. Thus, there is a need for a novel RRM algorithm addressing key issues, including:
\begin{itemize}
    \item Design for centralized management of enterprise WLAN, working as a part of Wireless LAN Controller;
    \item Applicability with current hardware and software, namely:
    \begin{itemize}
        \item Effortless integration with OpenWRT-based access points;
        \item Requires data like physical and link-layer statistics that can be obtained using only regular Linux Wi-Fi drivers like \texttt{nl80211} and standard Linux networking tools;
    \end{itemize}
    \item Performing not worse than existing RRM algorithm from Wimark Systems that will be analyzed in Chapter \ref{sec:baseline};
    \item Able to combine both channel selection and transmit power adjustment to improve RF environment and network performance.
\end{itemize}

The following chapters will focus on analyzing the limitations of current algorithms and developing a new one.