\chapter{Introduction}
\label{chap:intro}
\chaptermark{}

% indicate the importance of your topic
\section{Importance of the topic}

Nowadays, wireless local area networks (WLAN) implementing IEEE 802.11 standards, commonly known under "Wi-Fi" brand, become an increasingly popular solution for last-mile internet access with a diverse population of users, starting from home Wi-Fi routers up to large campus- and city-scale WLANs with coverage areas reaching several square kilometers.
As a result, the density of Wi-Fi network increases, so the frequency band allocated for 802.11 networks becomes more congested, which leads to interference and signal cancellation between different WLANs, resulting in network performance degradation.
Moreover, other appliances operating on frequencies that overlap with Wi-Fi band, undermining the performance of WLANs.
To meet current bandwidth and latency expectations of modern network applications, such as video streaming, cloud computing, and video conferencing, the wireless network must be able to provide sufficient capacity to all clients. In this light, proper radio resource management becomes crucial for operating wireless networks.

The problem of managing radio resources is studied extensively in the context of \textbf{cellular networks}, which are characterized by extensive frequencies reuse, large number of clients and large coverage areas spanning multiple kilometers, so the proper spectrum management is vital for operation of cells. This problem breaks down to the following: given a set of access points (or base stations in cellular terminology) $\boldsymbol{B}$, which can communicate over a set of channels $\boldsymbol{C}$, with a maximum transmit power of $P_{max}$, establish a radio link between a client device and an access point by assigning it a triplet $(b, c, p)$, where $b \in \boldsymbol{B}$, $c \in \boldsymbol{C}$, $p \leq P_{max}$.
Essentially, RRM algorithms aim to provide such assignments that maximize the overall network performance.
% indicate a lack or a gap in existing literature, possible limitations of the existing approaches
\section{Limitations of existing approaches}
A growing demand for Radio Resource Management (RRM) solutions, especially for large enterprise-grade multi-AP WLAN deployments, has led to the development of multiple commercial solutions, such as Cisco Radio Resource Management (RRM) \cite{ciscoRadioResourceManagement}, Aruba Adaptive Radio Management (ARM) \cite{UnderstandingARM}, and others. However, existing solutions offered by major vendors are proprietary, so their source code, used algorithms, and details of operation are not disclosed.
In the same time, most studies on RRM have only focused on cellular networks, while RRM in 802.11 networks received much less attention. Many existing works on RRM in 802.11 networks have applicability problems, since real-world hardware poses constraints on what metrics can be retrieved from the wireless interface and which physical and link-level parameters can be adjusted.

\section{Research gap}
Therefore, I can identify a research gap: the need for an open-source RRM (Radio Resource Management) algorithm that matches the performance of proprietary solutions and is feasible for real-world WLAN deployments.

This study aims to fill this gap by proposing a new RRM algorithm that, unlike existing market solutions, is publicly available, and at least as effective, while being practical and applicable to real-world WLAN deployments.

% indicate the contribution of your thesis project, establish its significance
\section{Contribution and significance of the study}
By analyzing theoretical works on 802.11 RRM, we come up with suitable optimization approach that combines both Adaptive Channel Selection (ACS) and Transmit Power Control (TPC) techniques to improve RF spectrum situation EM compatibility and channel reuse, which in turn leads to improved capacity of a wireless network.

Our RRM approach is encompassed within the centralized architecture, introduced by Cisco as Unified Wireless Network (UWN) \cite{CiscoUnifiedWirelessa}, a highly centralized wired-wireless architecture controlled by a Wireless LAN Controller (WLC). This approach allows for a more efficient spectrum management, as the WLC can collect data from all access points and make decisions based on the global view of the network, which is not possible in a distributed architecture, where each access point makes decisions independently.

% Outline of paper structure

The rest of this Thesis is structured as follows: Chapter \ref{chap:lr} reviews existing academic research on managing radio resources and publicly available information about proprietary RRM solutions; in Chapter 3, we formulate the mathematical model of transmission in a wireless network, review the existing RRM algorithm at Wimark Systems, identify its limitations and derive a new algorithm; in Chapter 4, we describe implementation details for the algorithm in NS-3 simulator and Wimark products; in Chapter 5, we evaluate the performance of our algorithm; Chapter 6 contains the results and discussion.

