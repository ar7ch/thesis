\chapter{Introduction}
\label{chap:intro}
\chaptermark{Optional running chapter heading}

Nowadays, wireless local area networks (WLAN) implementing IEEE 802.11 standards, which are commonly known under "Wi-Fi" brand, become an increasingly popular solution for last-mile internet access with a diverse population of users, starting from home Wi-Fi routers up to large campus- and city-scale WLANs with coverage areas reaching several square kilometers.
As a result, the density of Wi-Fi network increases, so the frequency band allocated for 802.11 networks becomes more congested, which leads to interference and signal cancellation between different WLANs, resulting in network performance degradation.
Moreover, other appliances operating on frequencies that overlap with Wi-Fi band, undermining the performance of WLANs.
This situation has led to a growing demand for Radio Resource Management (RRM) solutions, especially for large enterprise-grade multi-AP WLAN deployments. However, existing solutions offered by major vendors are proprietary, so their source code, used algorithms and details of operation are not disclosed.
In this Thesis, we present an implementation of a Radio Resource Management algorithm that is intended to be more effective then available predecessors while putting practical applicability in the first place.

Our approach benefits from the employment of Wireless LAN Controller (WLC) — a central entity in the WLAN architecture that is able to control and gather statistics from all the access points.
The rest of this Thesis is structured as follows: Chapter \ref{chap:lr} reviews existing academic research on managing radio resources and publicly available information about proprietary RRM solutions; in Chapter 3, we formulate the mathematical model of transmission in a wireless network, review the existing RRM algorithm at Wimark Systems, identify its weaknesses and derive a new algorithm; in Chapter 4, we describe implementation details for the algorithm in NS-3 simulator and Wimark products; in Chapter 5, we evaluate the performance of our algorithm; Chapter 6 contains the results and discussion.

Below in this chapter, we briefly review the fundamentals of Radio Frequency (RF) communications, and IEEE 802.11 standard for wireless LAN.

\section{Overview of IEEE 802.11 standard}

Throughout this Thesis, we will refer to field-specific terms, whose definitions are given in Table \ref{defs_table}:

\makeatletter
\let\@currsize\normalsize
\makeatother
\label{defs_table}
\begin{table}[h]
    \centering
    \begin{tabular}{| p{0.5\linewidth} | p{0.5\linewidth} |}\hline
        \textbf{Term} & \textbf{Definition} \\ \hline
        Signal & Airborne RF energy \\
        \hline
        Channel & A band of frequencies that 802.11 devices can use for communications \cite{AuthoritativeDictionaryIEEE2000} \\
        \hline
        Inteference & Destructive influence of another signal leading to degradation of signal quality and loss of frames \\
        \hline
        Noise & Signal that cannot be demodulated as 802.11 signal \\
        \hline
        SNR (Signal-to-Noise Ratio) & Signal quality metric, defined as ratio of signal power to the noise \\
        \hline
        SINR (Signal-to-Interference-plus-Noise Ratio) & b4 \\
        \hline
        RSSI (Received Signal Strength Indicator) & quality metric of a wireless signal \\
        \hline
        Radio cell & A geographical area covered by a radio transmitter \cite{tanenbaumComputerNetworks2020} \\
        \hline
        Basic Service Set (BSS) & A set of stations belonging to the same radio cell and exchanging information \cite{konsgenSpectrumManagementAlgorithms2010} \\
        \hline
        Distribution System (DS) & A network that interconnects multiple BSSs and provides connectivity to a wired network \cite{konsgenSpectrumManagementAlgorithms2010} \\
        \hline
        Extended Service Set (ESS) & A set of BSSs interconnected by a DS \cite{konsgenSpectrumManagementAlgorithms2010} \\
        \hline
        Infrastructure mode & A centralized mode of a 802.11 WLAN operation using a star topology, where AP serves as a central entity for managing WLAN and switching traffic \\
        \hline
        Station (STA) & A client device that can connect to a WLAN \\
        \hline
        Access Point (AP) & A device that provides wireless access to a wired network \\
        \hline
        Network capacity & Maximum transmission rate that any station can achieve in a given WLAN \\
        \hline
    \end{tabular}
    \caption{Used terms and definitions}
    \label{tab:my_label}
\end{table}

\subsection{Transmission medium}

The primary medium for communications in IEEE 802.11 are electromagnetic (EM) radio-frequency (RF) waves operating within the microwave range \cite{tanenbaumComputerNetworks2020}. Infrared radiation (IR) as a transmissions medium was defined in the legacy 802.11 standard and then had been deprecated; 802.11bb amendment, introducing communications through visible light, is yet to be finalized and become commercially mature technology.

\subsection{Frequency band}

Most of the 802.11 amendments, including b,g,n, and partially ax, operate at the unlicensed 2.4 GHz ISM (Industrial, Scientific, Medical) RF band \cite{tanenbaumComputerNetworks2020, colemanCWNACertifiedWireless2021}. Using an unlicensed frequency band, however, introduces multiple challenges: the radio spectrum becomes congested with non-802.11 sources, such as microwave ovens, Bluetooth Personal Area Networks (PAN), cordless phones etc. \cite{tanenbaumComputerNetworks2020, colemanCWNACertifiedWireless2021}. Moreover, 2.4 GHz signals can propagate through solid obstructions like walls, doors, and windows better than signals operating on higher-frequency ones \cite{colemanCWNACertifiedWireless2021}. This property can provide better coverage and signal quality for clients, although can cause interference for neighboring WLANs, which will in turn lead to degradation of signal quality.

2.4 GHz band is split into 14 channels, each 22 MHz wide [1]. Each channel is characterized by its \textit{center frequency}, +- 11 MHz, with 5 MHz width between two adjacent centers, i.e, the channels \textit{overlap}. Channel 1 has central frequency 2.412 GHz, Channel 14 — 2.484 GHz. Thus, for channels to be non-overlapping, they must have at least 5 channels in between. Such non-overlapping channels are 1, 6, 11, with central frequencies 2.412, 2.437, and 2.462 MHz, respectively.

Another frequency band is used by 802.11a/ac/ax radios is 5 GHz U-NII.

\subsection{Signal quality and its metrics}

Thus, the presence of physical obstructions, background noise and interference from other access points urges us to explore possible measurements and metrics for a wireless signal quality. Below, we will briefly describe the most widely used quantities:

A measure widely used in RF engineering and employed by Wi-Fi vendors is \textbf{Signal-to-Noise Ratio (SNR)}, which is defined as a ratio between the received signal power and the power of background noise [2]:
\begin{equation}
    \label{formula:snr}
    SNR = \frac{P_{signal}}{P_{noise}}
\end{equation}
    Since SNR is essentially a difference in power, which is measured in Watts, in practice it is measured in a relative unit on a logarithmic scale called \textbf{decibel (dB)} [1,2]:

\begin{equation}
    \label{formula:snr_db}
    {SNR}_{dB} = 10\log_{10}\frac{P_{signal}}{P_{noise}}
\end{equation}

In recent years, \textbf{Signal-to-Interference-Plus-Noise ratio (SINR)} measurement have become a more widespread measurement of wireless networks\' signal quality. Similarly, it is defined as:

\begin{equation}
    \label{formula:sinr}
    SINR = \frac{P_{signal}}{P_{noise} + P_{interf}}
\end{equation}

    where $P_{signal}$ is the power of the signal of interest, and $P_{interf}$ is the power of interfering signals.
    By considering interference from other 802.11 devices, which is typically a dynamic quantity that changes rapidly over time unlike background noise, SINR describes EM spectrum situation more accurately.

Received Signal Strength Indicator (RSSI) relative measure of signal strength in range from 0 to 255, where 0 is the weakest signal a receiver is able to sense. The exact correspondence between RSSI and received signal power is implementation-specific and is left on behalf of hardware manufacturers [1].

\subsection{Radio Resource Management}
\label{chap:intro:sec:rrm}
The scarcity of available frequency bands in the time of growing demand for wireless connectivity has led to the development of methods called \textit{spectrum management} or \textit{radio resource management (RRM)}. Most of research on RRM is focused on cellular networks, where coverage area of base stations spans across multiple kilometers, and the number of clients for one station can reach several thousands, so proper spectrum management is vital for operation of cells. However, from the physical layer perspective, the radio situation in 802.11 networks is similar. As described in \cite{zanderRadioResourceManagement1997}, given a wireless network with a set of access points $\boldsymbol{B}$, which can communicate over a set of channels $\boldsymbol{C}$, with a maximum transmit power of $P_{max}$, establishing a radio link between a client device and an access point requires from the wireless infrastructure to assign:
\begin{enumerate}
    \item An access point $b \in \boldsymbol{B}$;
    \item A frequency channel $c \in \boldsymbol{C}$;
    \item A transmission power level $ p \leq P_{max}$.
\end{enumerate}
Obviously, channel and transmission power are \textit{global} for a given access point in a sense that all its other clients will have to adjust their parameters correspondingly: switch the operating channel and deal with the new received signal strength from their AP.
In Wi-Fi, the first requirement is usually decided by the client itself: user chooses SSID they wish to use, and in case if multiple APs serve the same SSID, a client device associates with AP having the strongest signal available. Later, a client can switch to another access point within the same extended service set via \textit{roaming} methods, such as \textit{Fast Basic Service Set (BSS) Transition} defined in 802.11r \cite{80211r}. The roaming decision is ultimately made by a client device, which sends a reassociation request to start the roaming process \cite{colemanCWNACertifiedWireless2021}. The access point, however, can force a client to find another access point by sending a deauthentication frame, or moving to another channel without notifying.
The second and third requirements are a part of current AP configration and a subject to change. A client discovers current operating channel of APs by tuning on each available channel in a succession, while transmission power only can be estimated by measuring received signal strength.
Thus, \textbf{the goal of a radio resource allocation algorithm is to optimize spectrum usage within a WLAN via assigning an operating channel and a transmission power level to each access point in a way that maximizes the overall network performance.}

As it will be shown in Section \ref{chap:lr:sec:rrm_80211}, the 802.11 standard does not provide any algorithms for channel and transmission power assignment, however, some amendments introduce methods for measurement, signaling and radio adjustment that can be used for RRM purposes.

Note that related researches and commercial solutions introduce many different terms for the same procedure of channel change that can use different algorithms and slightly vary according to specifics of their application: \textit{Frequency Selection}, \textit{Frequency Planning}, \textit{Channel Selection}, \textit{Channel Planning}, \textit{Channel Assignment}, etc. Adjustment of transmission power is usually
In this Thesis, we consider those terms to be synonyms.