\chapter{Introduction}
\label{chap:intro}
\chaptermark{}

% indicate the importance of your topic
\section{Importance of the topic}

Wireless local area networks (WLANs) that implement the IEEE 802.11 standards, commonly known as Wi-Fi, have become an increasingly popular solution for last-mile Internet access with a diverse population of users, from home Wi-Fi routers to large campus and city-scale WLANs with coverage areas reaching several square kilometers.
As a result, the density of the Wi-Fi network increases, so the frequency band allocated for 802.11 networks becomes more congested, leading to interference and signal cancellation between different WLANs, resulting in network performance degradation.
In addition, other appliances, operating on frequencies that overlap with the Wi-Fi band, affect the performance of WLANs.
To meet the current bandwidth and latency expectations of modern network applications, such as video streaming, cloud computing, and video conferencing, the wireless network must be able to provide sufficient capacity to all clients. In this light, proper radio resource management becomes crucial for operating wireless networks.

The problem of managing radio resources is extensively studied in the context of \textbf{cellular networks}, which are characterized by extensive frequency reuse, a large number of clients, and large coverage areas spanning multiple kilometers, so proper spectrum management is vital for operation of cells. This problem is broken down to the following: given a set of access points (or base stations in cellular terminology) $\boldsymbol{B}$, which can communicate over a set of channels $\boldsymbol{C}$, with a maximum transmit power of $P_{max}$, establish a radio link between a client device and an access point by assigning it a triplet $(b, c, p)$, where $b \in \boldsymbol{B}$, $c \in \boldsymbol{C}$, $p \leq P_{max}$.
Essentially, RRM algorithms aim to provide such assignments that maximize the overall network performance.
% indicate a lack or a gap in existing literature, possible limitations of the existing approaches
\section{Limitations of existing approaches}
A growing demand for RRM solutions, especially for large enterprise-grade multi-AP WLAN deployments, has led to the development of multiple commercial solutions, such as Cisco RRM \cite{ciscoRadioResourceManagement}, Aruba Adaptive Radio Management (ARM) \cite{UnderstandingARM}, and others. However, existing solutions offered by major vendors are proprietary, so their source code, algorithms used, and operation details are not disclosed.
At the same time, most studies on RRM have focused only on cellular networks, while RRM in 802.11 networks received much less attention. Many existing works on RRM in 802.11 networks have applicability problems, since real-world hardware poses constraints on what metrics can be retrieved from the wireless interface and which physical and link-level parameters can be adjusted.

\section{Research gap}
Therefore, I can identify a research gap: the need for an open-source RRM algorithm that matches the performance of proprietary solutions and is feasible for real-world WLAN deployments.

This study aims to fill this gap by proposing a new RRM algorithm that, unlike existing market solutions, is publicly available and at least as effective while being practical and applicable to real-world WLAN deployments.

% indicate the contribution of your thesis project, establish its significance
\section{Contribution and significance of the study}
By analyzing theoretical works on 802.11 RRM, I came up with a suitable optimization approach that combines Adaptive Channel Selection (ACS) and Transmit Power Control (TPC) techniques to improve radio frequency (RF) spectrum utilization, and channel reuse, which in turn leads to improved capacity of a wireless network.

The RRM approach that I use adheres to the centralized architecture, introduced by Cisco as the Unified Wireless Network (UWN) \cite{CiscoUnifiedWirelessa}, a highly centralized wired-wireless architecture controlled by a Wireless LAN Controller (WLC). This approach allows for a more efficient spectrum management, as the WLC can collect data from all access points and make decisions based on the global view of the network, which is not possible in a distributed architecture, where each access point makes decisions independently.

% Outline of paper structure

The rest of this thesis is structured as follows: Chapter \ref{chap:lr} reviews existing academic research on managing radio resources and publicly available information about proprietary RRM solutions; in Chapter \ref{chap:research} I analyze \textit{RRMGreedy}, identify its limitations, and propose improvements that result in the new version of the algorithm called \textit{RRMGreedy++}; in Chapter \ref{chap:impl}, I present implementation and evaluation results using the ns-3 network simulator; Chapter \ref{chap:eval} contains the results and discussion.

