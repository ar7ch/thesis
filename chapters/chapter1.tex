\chapter{Introduction}
\label{chap:intro}
\chaptermark{}

% Write 2-3 connected sentences representative of each of the following four functions in Introduction chapter:
\section{Part 1. Introduction}
% indicate the importance of your topic
\subsection{Importance of the topic}
Nowadays, wireless local area networks (WLAN) implementing IEEE 802.11 standards, commonly known under "Wi-Fi" brand, become an increasingly popular solution for last-mile internet access with a diverse population of users, starting from home Wi-Fi routers up to large campus- and city-scale WLANs with coverage areas reaching several square kilometers.
As a result, the density of Wi-Fi network increases, so the frequency band allocated for 802.11 networks becomes more congested, which leads to interference and signal cancellation between different WLANs, resulting in network performance degradation.
Moreover, other appliances operating on frequencies that overlap with Wi-Fi band, undermining the performance of WLANs.
To meet current bandwidth and latency expectations of modern network applications, such as video streaming, cloud computing, and video conferencing, the wireless network must be able to provide sufficient capacity to all clients. In this light, proper radio resource management becomes crucial for operating wireless networks.

The problem of managing radio resources is studied extensively in the context of \textbf{cellular networks}, which are characterized by extensive frequencies reuse, large number of clients and large coverage areas spanning multiple kilometers, so the proper spectrum management is vital for operation of cells. This problem breaks down to the following: given a set of access points (or base stations in cellular terminology) $\boldsymbol{B}$, which can communicate over a set of channels $\boldsymbol{C}$, with a maximum transmit power of $P_{max}$, establish a radio link between a client device and an access point by assigning it a triplet $(b, c, p)$, where $b \in \boldsymbol{B}$, $c \in \boldsymbol{C}$, $p \leq P_{max}$.
Essentially, RRM algorithms aim to provide such assignments that maximize the overall network performance.
% indicate a lack or a gap in existing literature, possible limitations of the existing approaches
\subsection{Limitations of existing approaches}
A growing demand for Radio Resource Management (RRM) solutions, especially for large enterprise-grade multi-AP WLAN deployments, has led to the development of multiple commercial solutions, such as Cisco Radio Resource Management (RRM) \cite{ciscoRadioResourceManagement}, Aruba Adaptive Radio Management (ARM) \cite{UnderstandingARM}, and others. However, existing solutions offered by major vendors are proprietary, so their source code, used algorithms and details of operation are not disclosed.
In the same time, most studies on RRM have only focused on cellular networks, while RRM in 802.11 networks received much less attention. Many existing works on RRM in 802.11 networks have applicability problems, since real-world hardware poses constraints on what metrics can be retrieved from the wireless interface and which physical and link-level parameters can be adjusted.
% indicate the contribution of your thesis project, establish its significance
\subsection{Contribution and significance of the thesis}
This study aims to fill this gap by proposing a new RRM algorithm that, unlike existing market solutions, is publicly available, and at least as effective, while being practical and applicable to real-world WLAN deployments.
By analyzing theoretical works on 802.11 RRM, we come up with suitable optimization approach that combines both Adaptive Channel Selection (ACS) and Transmit Power Control (TPC) techniques to improve RF spectrum situation EM compatibility and channel reuse, which in turn leads to improved capacity of a wireless network.

Our RRM approach is encompassed within the centralized architecture, introduced by Cisco as Unified Wireless Network (UWN) \cite{CiscoUnifiedWirelessa}, a highly centralized wired-wireless architecture controlled by a Wireless LAN Controller (WLC). This approach allows for a more efficient spectrum management, as the WLC can collect data from all access points and make decisions based on the global view of the network, which is not possible in a distributed architecture, where each access point makes decisions independently.


% Outline of paper structure

The rest of this Thesis is structured as follows: Chapter \ref{chap:lr} reviews existing academic research on managing radio resources and publicly available information about proprietary RRM solutions; in Chapter 3, we formulate the mathematical model of transmission in a wireless network, review the existing RRM algorithm at Wimark Systems, identify its limitations and derive a new algorithm; in Chapter 4, we describe implementation details for the algorithm in NS-3 simulator and Wimark products; in Chapter 5, we evaluate the performance of our algorithm; Chapter 6 contains the results and discussion.



Below in this chapter, we briefly review the fundamentals of Radio Frequency (RF) communications, and IEEE 802.11 standard for wireless LAN.

\section{Overview of IEEE 802.11 standard}

Throughout this Thesis, we will refer to field-specific terms, whose definitions are given in Table \ref{defs_table}:

\makeatletter
\let\@currsize\normalsize
\makeatother
\label{defs_table}
\begin{table}[h]
    \centering
    \begin{tabular}{| p{0.5\linewidth} | p{0.5\linewidth} |}\hline
        \textbf{Term} & \textbf{Definition} \\ \hline
        Signal & Airborne RF energy \\
        \hline
        Channel & A band of frequencies that 802.11 devices can use for communications \cite{AuthoritativeDictionaryIEEE2000} \\
        \hline
        Inteference & Destructive influence of another signal leading to degradation of signal quality and loss of frames \\
        \hline
        Noise & Signal that cannot be demodulated as 802.11 signal \\
        \hline
        SNR (Signal-to-Noise Ratio) & Signal quality metric, defined as ratio of signal power to the noise \\
        \hline
        SINR (Signal-to-Interference-plus-Noise Ratio) & Signal quality metric, defined as ratio of signal power to the sum of noise power and power of interefering signals \\
        \hline
        RSSI (Received Signal Strength Indicator) & A metric of a wireless signal quality defined in 802.11, varying from 0 to 255, where exact mappings to Rx signal power are vendor-specific. \\
        \hline
        Radio cell & A geographical area covered by a radio transmitter \cite{tanenbaumComputerNetworks2020} \\
        \hline
        Basic Service Set (BSS) & A set of stations belonging to the same radio cell and exchanging information \cite{konsgenSpectrumManagementAlgorithms2010} \\
        \hline
        Distribution System (DS) & A network that interconnects multiple BSSs and provides connectivity to a wired network \cite{konsgenSpectrumManagementAlgorithms2010} \\
        \hline
        Extended Service Set (ESS) & A set of BSSs interconnected by a DS \cite{konsgenSpectrumManagementAlgorithms2010} \\
        \hline
        Infrastructure mode & A centralized mode of a 802.11 WLAN operation using a star topology, where AP serves as a central entity for managing WLAN and switching traffic \\
        \hline
        Station (STA) & A client device that can connect to a WLAN \\
        \hline
        Access Point (AP) & A device that provides wireless access to a wired network \\
        \hline
        Network capacity & Maximum transmission rate that any station can achieve in a given WLAN \\
        \hline
    \end{tabular}
    \caption{Used terms and definitions}
    \label{tab:my_label}
\end{table}

\subsection{Transmission medium}

The primary medium for communications in IEEE 802.11 are electromagnetic (EM) radio-frequency (RF) waves operating within the microwave range \cite{colemanCWNACertifiedWireless2021}.

\subsection{Frequency band}
Main frequency bands are: 2.4 GHz band introduced by 802.11b, 5 GHz band introduced by 802.11a and 6 GHz, introduced by 802.11ax. This study focuses on first two bands as used the most frequently. Below I describe those bands in more details.

Most of the 802.11 amendments, including b,g,n, and partially ax, operate at unlicensed 2.4 GHz ISM (Industrial, Scientific, Medical) RF band \cite{tanenbaumComputerNetworks2020, colemanCWNACertifiedWireless2021}. Using an unlicensed frequency band, however, introduces multiple challenges: the radio spectrum becomes congested with non-802.11 sources, such as microwave ovens, Bluetooth Personal Area Networks (PAN), cordless phones etc. \cite{tanenbaumComputerNetworks2020, colemanCWNACertifiedWireless2021}. Moreover, 2.4 GHz signals can propagate through solid obstructions like walls, doors, and windows better than signals operating on higher-frequency ones \cite{colemanCWNACertifiedWireless2021}. This property can provide better coverage and signal quality for clients, although can cause interference for neighboring WLANs, which will in turn lead to degradation of signal quality.

The 2.4 GHz ISM band is split into 14 channels. Depending on local regulations, number of possible channels can vary, but in general channels 1-11 are available at every region. Assuming 20 MHz channel width, each channel is characterized by its \textit{center frequency}, $\pm$ 10 MHz, with 5 MHz width between two adjacent centers, i.e, the channels \textit{overlap}. Channel 1 has central frequency 2.412 GHz, Channel 14 — 2.484 GHz. Thus, for channels to be non-overlapping, they must have at least 5 channels or 25 MHz in between. Such non-overlapping channels are 1, 6, 11, with central frequencies 2.412, 2.437, and 2.462 MHz, respectively.

The 5 GHz U-NII (Unlicensed National Information Infrastructure) series of bands is used by 802.11a/802.11ac/802.11ax amendments. Unlike the 2.4 GHz band, channels do not overlap. On the end of each band \textit{guard band} as an additional measure to avoid interference. Combined, bands U-NII-1, U-NII-1, U-NII-2A, U-NII-2C, U-NII-3 provide twenty-five 20 MHz or twelve 40 MHz non-overlapping channels \cite{colemanCWNACertifiedWireless2021}. However, some channels may not be available in different regions, since this band can also be used by military and weather radars. The first channel from U-NII-1 band has number 36.


\subsection{Signal quality and its metrics}

Thus, the presence of physical obstructions, background noise and interference from other access points urges us to explore possible measurements and metrics for a wireless signal quality. Below, we will briefly describe the most widely used quantities:

A measure widely used in RF engineering and employed by Wi-Fi vendors is \textbf{Signal-to-Noise Ratio (SNR)}, which is defined as a ratio between the received signal power and the power of background noise:
\begin{equation}
    \label{formula:snr}
    SNR = \frac{P_{signal}}{P_{noise}}
\end{equation}
    Since SNR is essentially a difference in power, which is measured in Watts, in practice it is measured in a relative unit on a logarithmic scale called \textbf{decibel (dB)} [1,2]:

\begin{equation}
    \label{formula:snr_db}
    {SNR}_{dB} = 10\log_{10}\frac{P_{signal}}{P_{noise}}
\end{equation}

In recent years, \textbf{Signal-to-Interference-Plus-Noise ratio (SINR)} measurement have become a more widespread measurement of wireless networks\' signal quality. Similarly, it is defined as:

\begin{equation}
    \label{formula:sinr}
    SINR = \frac{P_{signal}}{P_{noise} + P_{interf}}
\end{equation}

    where $P_{signal}$ is the power of the signal of interest, and $P_{interf}$ is the power of interfering signals.
    By considering interference from other 802.11 devices, which is typically a dynamic quantity that changes rapidly over time unlike background noise, SINR describes EM spectrum situation more accurately.

Received Signal Strength Indicator (RSSI) relative measure of signal strength in range from 0 to 255, where 0 is the weakest signal a receiver is able to sense. The exact correspondence between RSSI and received signal power is implementation-specific and is left on behalf of hardware manufacturers \cite{colemanCWNACertifiedWireless2021}.

\subsection{Radio Resource Management}
\label{chap:intro:sec:rrm}
The scarcity of available frequency bands in the time of growing demand for wireless connectivity has led to the development of methods called \textit{spectrum management} or \textit{radio resource management (RRM)}. Most of research on RRM is focused on cellular networks, where coverage area of base stations spans across multiple kilometers, and the number of clients for one station can reach several thousands, so proper spectrum management is vital for operation of cells. However, from the physical layer perspective, the radio situation in 802.11 networks is similar. As described in \cite{zanderRadioResourceManagement1997}, given a wireless network with a set of access points $\boldsymbol{B}$, which can communicate over a set of channels $\boldsymbol{C}$, with a maximum transmit power of $P_{max}$, establishing a radio link between a client device and an access point requires from the wireless infrastructure to assign:
\begin{enumerate}
    \item An access point $b \in \boldsymbol{B}$;
    \item A frequency channel $c \in \boldsymbol{C}$;
    \item A transmission power level $ p \leq P_{max}$.
\end{enumerate}
Obviously, channel and transmission power are \textit{global} for a given access point in a sense that all its other clients will have to adjust their parameters correspondingly: switch the operating channel and deal with the new received signal strength from their AP.
In Wi-Fi, the first requirement is usually managed by the client itself: user chooses SSID they wish to use, and in case if multiple APs serve the same SSID, a client device associates with AP having the strongest signal available. Later, a client can switch to another access point within the same extended service set via \textit{roaming} methods, such as \textit{Fast Basic Service Set (BSS) Transition} defined in 802.11r \cite{80211r2008IEEE}. The roaming decision is ultimately made by a client device, which sends a reassociation request to start the roaming process \cite{colemanCWNACertifiedWireless2021}. The access point, however, can force a client to find another access point by sending a deauthentication frame, or moving to another channel without notifying.
The second and third requirements are a part of current AP configuration and a subject to change. A client discovers current operating channel of APs by tuning on each available channel in a succession, while transmission power only can be estimated by measuring received signal strength.
Thus, \textbf{the goal of a radio resource allocation algorithm is to optimize spectrum usage within a WLAN via assigning an operating channel and a transmission power level to each access point in a way that maximizes the overall network performance.}

As it will be shown in Section \ref{chap:lr:sec:rrm_80211}, the 802.11 standard does not provide any algorithms for channel and transmission power assignment, however, some amendments introduce methods for measurement, signaling and radio adjustment that can be used for RRM purposes.

Note that related researches and commercial solutions introduce many similar terms for the same procedure of channel change that can use different algorithms and slightly vary according to specifics of their application: \textit{Frequency Selection}, \textit{Frequency Planning}, \textit{Channel Selection}, \textit{Channel Planning}, \textit{Channel Assignment}, etc. Adjustment of transmission power is usually
In this study, we use those terms interchangeably.