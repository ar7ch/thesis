\newcommand{\rrmG}{$\boldsymbol{G}$}

\chapter{Baseline solution analysis and algorithm design}
\label{chap:research}

The chapter is organized as follows. Section \ref{chap:research:sec:problem_statement} defines the problem of RRM and the objectives of an RRM algorithm.
In Section \ref{sec:baseline}, I describe the RRM algorithm called \textit{RRMGreedy}, developed by a Russian telecommunications vendor, analyze the flow and asymptotic complexity of the algorithm and identify its flaws and limitations. The objective of this study is to create an algorithm outperforming \textit{RRMGreedy} in terms of network capacity and throughput; in Section \ref{chap:research:sec:rrmv2} I propose a new RRM algorithm, called \textit{RRMGreedy++}, that addresses the flaws of \textit{RRMGreedy}.


\section{Problem statement}
\label{chap:research:sec:problem_statement}
Using the definition of the RRM problem from \ref{chap:lr:sec:math_models}, I define the problem as follows: given a set of access points $AP = \{AP_1, AP_2, \ldots, AP_n\}$, where each access point $AP_i$ has a set of wireless interfaces $W_i = \{w_{i1}, w_{i2}, \ldots, w_{im}\}$, where each interface $w_{ij}$ can operate on a set of channels $C = \{c_1, c_2, \ldots, c_k\}$, the goal is to find a channel assignment $f: W \to C$ and transmit power assignment $g: W \to P$ that maximizes network throughput and capacity. The problem can be formulated as a combinatorial optimization problem, where the goal is to find a configuration of channels and transmit powers that maximizes the objective function, which is a function of network throughput and capacity.

\section{Baseline solution analysis --- RRMGreedy}
\label{sec:baseline}
The greedy RRM algorithm, used by a major telecommunications vendor, which will be further referred to as \textit{RRMGreedy}, is based on background scanning by access points. The goal of this study is to create a new RRM algorithm that complies with the requirements and addresses the research gap formulated in Section \ref{chap:lr:sec:conclusion}.
In particular, the novel algorithm should result in network throughput and capacity greater or equal than \textit{RRMGreedy} in corner cases. In this light, I start with describing \textit{RRMGreedy}.

\subsection{RRMGreedy: flow description}
This section describes \textit{RRMGreedy}.
As the name suggests, the algorithm tries to achieve optimal radio resource allocation in a greedy way, taking the local optimum for each device in an RRM group.
The algorithm operates on the granularity of interfaces rather than access points. I consider it to be reasonable, since practically different interfaces operate on completely different frequency bands, so when are dealing with, for example, the 5 GHz band, the whole access point can be considered as a single interface.

In other words, I consider \textbf{RRM group} $G$ as a set of \textbf{wireless interfaces} $w$ (which will be further referred to as simply \textbf{interfaces}), where several interfaces can belong to a single access point.
On a high level, the algorithm consists of two main steps --- \textbf{Channel Selection} and \textbf{Transmit Power Adjustment}:
\everymath{\displaystyle}
\begin{enumerate}
\item \textbf{Channel Selection}:
    \begin{enumerate}
        \item \label{gi} compute \textbf{group interference} $I_{G}$ --- sum of interference scores for each interface in group:
            \begin{equation}
                I_{G} = \sum_{i=0}^{\lvert G \rvert} OnIfaceInterference(w_i)
            \end{equation}
        Initial group interference score is denoted as $I_{0}$.
        \item for each $w_i$, compute its updated interference score for every channel $c^k_i$ available for that interface:
            \begin{equation}
                interf_i = OnIfaceInterference(w_i, c^k_i)
            \end{equation}
        If there is a channel $c^k_i$ that reduced $interf_i$, update interface settings to issue change to that channel later.
        \item \textbf{continue while there is an improvement}, i.e., previous group interference score is larger than the latest one:
        $I_{j-1} > I_{j}$ for $j > 1$
    \end{enumerate}
\item \textbf{Transmit Power Adjustment}:
    \begin{enumerate}
        \item Compute group interference as in step \ref{gi}
        \item Identify the \textit{worst interface} $w_m$ causing the \textit{worst interference} (i.e., whose from-interference score is the largest):
            \begin{equation}
                w_m = \argmax_i FromIfaceInterference(w_i)
            \end{equation}
        \item Gradually reduce transmit power of $w_m$ with exponential backoff until it stops being the worst interface or reaches minimum Tx power.
    \end{enumerate}
\end{enumerate}

\renewcommand{\algorithmcfname}{ALGORITHM}
\SetAlFnt{\small}
\SetAlCapFnt{\small}
\SetAlCapNameFnt{\small}
\SetAlCapHSkip{0pt}
\IncMargin{-\parindent}
\LinesNumbered
\begin{algorithm}[hbpt]
\caption{RRMGreedy}
\label{alg:greedyRRM}

\SetKwInput{KwInput}{Input}
\SetKwInput{KwOutput}{Output}
\SetKwInput{KwData}{Data}
\SetKwComment{Comment}{$\triangleright$\ }{}
\SetKwFor{For}{for}{do}{endfor}
\SetKwIF{If}{ElseIf}{Else}{if}{then}{else if}{else}{endif}
\SetKwRepeat{Repeat}{repeat}{until}

\DontPrintSemicolon

\KwInput{
    $\boldsymbol{scandata}$: scanning data from access points \newline
    $\boldsymbol{G}$: set of wireless interfaces within the RRM group and its settings
}
\KwOutput{$\boldsymbol{resultConfig}$: final configuration for access points}

\BlankLine
groupState $\gets$ PreprocessRRMData($\boldsymbol{G}$, scandata)\;
\BlankLine
\tcp{Step 1: Start with max Tx power for each interface}
\For{each interface in $\boldsymbol{G}$}{
    groupState[interface].TxDiff $\gets (interface.MaxTxPower - interface.TxPower)$\;
}
\BlankLine
\tcp{Step 2: Greedy Channel Search}
groupInterference $\gets$ GroupInterference(groupState)\;
\Repeat{no improvement in groupInterference}{
    \For{each interface in $\boldsymbol{G}$}{
        $\boldsymbol{C_i} \gets$ available channels for interface\;
        newChannel $\gets \argmin_{ch \in \boldsymbol{C_i}} OnIfaceInterference(groupState,interface, ch)$\;
        groupState[interface].channel $\gets$ newChannel\;
    }
    groupInterference $\gets$ GroupInterference(scandata)\;
}
\BlankLine
\tcp{Step 3: Greedy Transmission Power Adjustment (if enabled)}
\Repeat{no improvement in groupInterference}{
    \tcp{Identify the "worst interface" $interface_{worst}$ causing the greatest interference for other APs and the maximum RSSI of $interface_{worst}$ on some other interface}
    $interface_{worst},\;RSSI_{worst}$ $\gets \argmax_{w_i \in G} FromIfaceInterference(groupState, w_i)$\;
    \tcp{$RSSI_{min}$ is a predefined threshold for the minimum RSSI, by default set to -100 (dBm)}
    $groupState[interface_{worst}].TxDiff \gets \frac{RSSI_{min} - RSSI_{worst}}{2}$ \;
    groupInterference $\gets$ GroupInterference(scandata)\;
}
\BlankLine
\tcp{Step 4: Send configuration changes to APs}
$\boldsymbol{resultConfig} \gets$ getUpdatedConfiguration(groupState)\;
\Return{$\boldsymbol{resultConfig}$}\;
\end{algorithm}

After providing a high-level overview of the algorithm, I will analyze the used functions, since they contain the core logic of the algorithm. These functions will be discussed in the order they are executed within the algorithm.

Function \textit{PreprocessRRMData()} does not perform actual RRM logic and is used to preprocess the data for further calculations. Namely, it reorganizes scan results for each interface in a way that for every channel every interface maintains a list of
\begin{itemize}
    \item Signals originating from access points within the RRM group \rrmG. In RRMGreedy, such signals are called \textit{inner} signals;
    \item Signals originating from access points outside of the RRM group (i.e., coming from \textit{rogue APs}). In RRMGreedy terms, such signals are called \textit{outer} signals.
\end{itemize}
The pseudocode for \textit{PreprocessRRMData()} is shown in Algorithm \ref{alg:processScanDataForGroup}.

\begin{algorithm}[H]
\label{alg:processScanDataForGroup}
\caption{PreprocessRRMData}
\DontPrintSemicolon

\ForEach{interface in \rrmG}{
    scanFromIface $\gets$ scandata[interface]\;
    \ForEach{BSSID, signalData in scanFromIface}{
        channel $\gets$ signalData.channel\;
        \eIf{bssid in RRMGroup}{
            interface.InnerBSS[channel].append(signalData)\;
        }{
            interface.OuterBSS[channel].append(signalData)\;
        }
    }
}
\end{algorithm}
The result of PreprocessRRMData is saved in a table called \textit{groupState}. This structure maintains updated channel and transmit power settings for each interface in the RRM group as RRMGreedy progresses.


% Algorithm \ref{lst:interf_on_cpe} is a pseudocode for 
Function \textit{OnIfaceInterference(groupState, interface, channel)} (Algorithm \ref{lst:channel_interf}) calculates interference experienced by given access point (and thus by its wireless interface) from outer and inner signals.
Essentially, it breaks down to calculating interference as a sum of outer and inner interference scores for a given interface $w_i$ on a given channel $c_i$:
% where:
% \begin{itemize}
%     \item $S^{outer}_i$ is a set of signals sensed by $w_i$ on from stations not in $G$;
%     \item $S^{inner}_i$ is a set of signals sensed by $w_i$ from stations in $G$, $s^j_i$ is $j$-th signal in $S^{outer}_i$ or $S^{inner}_i$
%     \item $scale(x)$ is a min-max normalization function that normalizes $x$ to [0, 1] scale with given $maxSignal$ and $minSignal$:
%         \begin{equation} scale(x) = \frac{x - minSignal}{maxSignal - minSignal} \end{equation}
%     \item $w_i.TxDiff$ is a transmit power adjustment for $w_i$;
% \end{itemize}
\newenvironment{conditions*}
  {\par\vspace{\abovedisplayskip}\noindent
   \tabularx{\columnwidth}{>{$}l<{$} @{}>{${}}c<{{}$}@{} >{\raggedright\arraybackslash}X}}
  {\endtabularx\par\vspace{\belowdisplayskip}}


\newenvironment{conditions}
  {\par\vspace{\abovedisplayskip}\noindent
   \begin{tabular}{>{$}l<{$} @{} >{${}}c<{{}$} @{} l}}
  {\end{tabular}\par\vspace{\belowdisplayskip}}

\begin{equation}
    interf_i = OnIfaceInterference(w_i, channel) = interf^{outer}_i + interf^{inner}_i
\end{equation}
\begin{equation}
    interf^{outer}_i = \sum_{j=0}^{\lvert S^{outer}_i \rvert} ChannelInterference(channel, s^{j}_i.channel) \cdot scale(s^j_i)
\end{equation}
\begin{equation}
    interf^{inner}_i = \sum_{j=0}^{\lvert S^{inner}_i \rvert} ChannelInterference(channel, s^{j}_i.channel) \cdot scale(s^j_i + w_i.TxDiff)
\end{equation}
where:
\begin{conditions*}
    S^{outer}_i &  & set of signals sensed by $w$ on from APs $\notin G$; \\
    S^{inner}_i &  & set of signals sensed by $w_i$ from APs $\in G$; \\
    s^j_i       &  & $j$-th signal received by $w_i$, in $S^{outer}_i$ or $S^{inner}_i$; \\
    w_i.TxDiff  &  & transmit power adjustment for $w_i$; \\
    ChannelInterference(ch_1, ch_2)  & \; \; & interference between $ch_1$ and $ch_2$ \linebreak (Algorithm \ref{lst:channel_interf}); \\
    scale(x)    &  & normalization function \ref{eq:scale}. \\
\end{conditions*}

Transmit power is normalized to $[0,1]$ with min-max normalization using pre-defined RSSI (Received Signal Strength Indicator) thresholds $RSSI_{min}$ and $RSSI_{max}$ (\ref{eq:scale}):
\begin{equation}
    \label{eq:scale}
    scale(x) = \frac{x - RSSI_{min}}{RSSI_{max} - RSSI_{min}}
\end{equation}

The interference between channels is estimated with \textit{ChannelInterference()} function(Algorithm \ref{lst:channel_interf}). The distance between adjacent channels is assumed to be 5 MHz. The interval of length $n$ should have $n-1$ channels in between, which is the reason for the factor $add$.


% In other words, the algorithm is as follows:

% \begin{enumerate}
%     \item \textbf{Calculate outer interference}: as defined earlier in this chapter, \textbf{outer interference} is interference originating from access points not in the RRM group $G$.
%     Based on earlier measurements sent by $w_i$, for each sensed signal $s \in S_i = S^{inner}_i \cap S^{outer}_i$, calculate adjacent-channel/co-channel interference score with \textit{ChannelInterference()}. This function, given current interface channel, its channel width, and some other channel, returns an estimation of interference between these channels using a simple heuristic (Listing \ref{lst:channel_interf}). Then, for each signal on a given channel, calculate interference score as a product of \textit{ChannelInterference()} and signal power normalized to [0, 1] scale; finally, sum up interference scores for all signals on this channel:
%     \begin{equation}
%         interf^{outer}_i = \sum_{j=0}^{\lvert S^{outer}_i \rvert} ChannelInterference(w_i, s^j_i) \cdot \frac{signal^j_i - minSignal}{maxSignal - minSignal}
%     \end{equation}
%     where $S^{outer}_i$ is a set of signals sensed by $w_i$ on channels not in $\mathbf{G}$, $s^j_i$ is $j$-th signal in $S^{outer}_i$, $signal^j_i$ is power of $s^j_i$ normalized to [0, 1] scale, and $maxSignal$ is the maximum signal power in $S^{outer}_i$.
%     \item \textbf{Calculate inner interference}: as defined earlier in this chapter, \textbf{inner interference} is interference originating from access points in the RRM group $G$. In the same way as for outer interference, calculate interference score for each signal on given channel as a product of \textit{ChannelInterference()} and signal power normalized to [0, 1] scale. One difference is that TxDiff is taken into account for Tx power adjustment; finally, sum up interference scores for all signals on this channel:
%     \begin{equation}
%         interf^{inner}_i = \sum_{j=0}^{\lvert S^{inner}_i \rvert} ChannelInterference(w_i, s^j_i) \cdot \frac{signal^j_i + w_i.TxDiff}{maxSignal}
%     \end{equation}
%     where $S^{inner}_i$ is a set of signals sensed by $w_i$ on channels in $G$, $s^j_i$ is $j$-th signal in $S^{inner}_i$, $signal^j_i$ is power of $s^j_i$ normalized to [0, 1] scale, and $maxSignal$ is the maximum signal power in $S^{inner}_i$.
% \end{enumerate}

\begin{algorithm}[H]
\caption{ChannelInterference}
\label{lst:channel_interf}
\DontPrintSemicolon % Don't print semicolons at the end of each line
\SetKwFunction{FMain}{ChannelInterference}
\SetKwProg{Fn}{Function}{:}{}
\Fn{\FMain{$ch_1$, $ch_2$, width}}{
    add $\gets$ 0\;
    \If{$ch_1 < 36$}{ % channels less than 36 belong to 2.4 GHz band
        add $\gets$ 1\;
    }
    \eIf{$\card{ch_1 - ch_2} \geq \frac{width}{5} + add$}{
        \Return 0  % No significant interference
    }{
        \Return 1  % Significant interference
    }
}
\end{algorithm}

\begin{algorithm}[H]
\label{lst:onifaceinterf}
\caption{OnIfaceInterference}
\DontPrintSemicolon
\KwIn{
    $groupState$: table containing state of \rrmG \newline
    $interface$: interface to calculate interference for
    $channel$: channel to calculate interference on
}
\KwOut{Estimated cumulative interference score experienced by $interface$}

$ifaceScanData \gets scandata[interface]$\;
$cumOuterInterf \gets 0$\;

\tcp{Calculate interference from outer APs}
\ForEach{otherChannel, otherChannelSignals in ifaceScanData.OuterBSS}{
    $channelInterfScore \gets ChannelInterference(otherChannel, interface.Channel, interface.ChannelWidth)$\;
    $channelOuterInterf \gets 0$\;
    \ForEach{signal in otherChannelSignals}{
        $sigRating \gets$ signal.RxPower (in dBm) normalized to $[0, 1]$ scale\;
        $channelOuterInterf \gets channelOuterInter + finterferenceScore \times sigRating$\;
    }
    $cumOuterInterf \gets cumOuterInterf + channelOuterInterf$\;
}
$cumInnerInterf \gets 0$\;
\tcp{Calculate interference from inner APs}
\ForEach{otherInterface, signalsFromOtherInterface in ifaceScanData.InnerBSS}{
    $channelInterfScore \gets ChannelInterference(oth.Settings.Central, interfaceChannel, interfaceChannelWidth)$\;
    \If{$channelInterfScore == 0$}{
        continue\;
    }
    $channelOuterInterference \gets 0$\;
    \ForEach{signal in signalsFromOtherInterface}{
        $adjustedSourceSignal \gets signal + otherInterface.TxDiff$\;
        $sigRating \gets adjustedSourceSignal$ power (in dBm) normalized to $[0, 1]$ scale\;
        $channelOuterInterference \gets channelInterfScore \times sigRating$\;
    }
    $cumInnerInterf \gets cumInnerInterf + channelOuterInterf$\;
}

\Return $cumOuterInterf + cumInnerInterf$\;
\end{algorithm}

% \subsection{Transmit Power Control}

The function \textit{FromIfaceInterference()} (Algorithm \ref{rrm:fromifaceinterf}) calculates the amount of interference originating from $w_i$ and experienced by other interfaces in the group \rrmG. The function returns two values: the cumulative interference quantity, that is, how much $w_i$ affects the RRM group, and the maximum RSSI of $w_i$ on some other interface $w_j$. Using \textit{FromIfaceInterference()}, RRMGreedy finds the \textit{worst interference} and gradually reduces its transmit power to improve the group interference score.

% The RRMGreedy TPC works as follows:
% \begin{enumerate}
%     \item check all interfaces $w_i \in G$.
%     \item if $w_i$ interferes with currently considered interface $w, w \ne w_i$, all measurements of $w$ on $w_i$ are computed as a cumulative factor, and maximum RSSI of $w$ on some other interface $w_j$ is saved.
%     \item Then, find the \textit{worst interface}, i.e., the one causing the most interference on others. To reduce Worst interface's transmit power should is reduced nearly in two, to the nearest defined power value, . Then group interference is recalculated in an iterative way while there is an improvement.
% \end{enumerate}
% Algorithm \ref{rrm:fromifaceinterf} demonstrates pseudocode for this function.
\newcommand{\rrmIface}{interface}
\newcommand{\rrmScandata}{scandata}

\begin{algorithm}[hbpt]
\caption{FromIfaceInterference()}
\label{rrm:fromifaceinterf}
\DontPrintSemicolon
\KwIn{
    $groupState$: table containing state of \rrmG \newline
    $interface$: interface to calculate interference from
}
\KwOut{
    $cumInterfFromInterface$: cumulative interference quantity estimating how much $\rrmIface$ affects the RRM group \newline
    $maxSignal$: maximum signal strength from $\rrmIface$ experienced by other AP}

$cumInterfFromInterface \gets 0$\;
$maxSignal \gets MinSignal$\;
\ForEach{otherInterface in scandata}{
    \If{otherInterface == interface}{
        continue\;
    }
    \If{otherInterface has not received signals from $\rrmIface$}{
        continue\;
    }
    $interfScore \gets ChannelInterference(\rrmIface.Channel, otherInterface.channel, otherInterface.channelWidth)$\;
    \If{$interfScore == 0$}{
        continue\;
    }
    $interfaceInterfSum \gets 0$\;
    \ForEach{sig in thisIface}{
        $adjustedSignalPower \gets \rrmIface.Signal + \rrmIface.TxDiff$\;
        $maxsignal \gets \max(signal, maxsignal)$\;
        $sigRating \gets$ adjustedSignalPower normalized to [0, 1]\;
        $interfaceInterfSum \gets interfaceInterfSum + (interfScore \times sigRating)$\;
    }
    $cumInterfFromInterface \gets cumInterfFromInterface + interfaceInterfSum$\;
}
\Return $cumInterfFromInterface, maxsignal$\;
\end{algorithm}

\subsection{Evaluating asymptotic complexity of RRMGreedy}
\label{chapter:research:sec:rrmgreedy:subsec:complexity}
In this section, I evaluate the asymptotic complexity of \textit{RRMGreedy} algorithm.
First, I consider the complexity of used auxiliary functions:
\begin{enumerate}
    \item $ChannelInterference(ch_1, ch_2, width)$ has constant complexity $O(1)$;
    \item $OnIfaceInterference(G, w_i)$ has complexity linear w.r.t. the number of signals sensed by $w_i$, i.e.,
        $O(\card{S_i}) = O(\card{S^{outer}_i} + \card{S^{inner}_i})$;
    \item $FromIfaceInterference(G, w_i)$ has complexity proportional to the product of number of interfaces in $G$ and amount of signal samples from $w_i$ heard by other interfaces, which is limited by some constant $C$, reasonably small in practice ($C < 10$), i.e.,
        $O(\card{G} \cdot C) = O(\card{G})$.
\end{enumerate}

Next, I analyze the time complexity of the main phases of the algorithm:

\begin{enumerate}
    \item \textbf{ProcessScanDataForGroup} has complexity $O(\card{G} \cdot \lvert S \rvert)$, where $\lvert S \rvert$ is the total number of signals in all scans from all interfaces in $G$;
    \item \textbf{Set Maximum Transmission Power} has complexity $O(\card{G})$;
    \item \textbf{Optimize Channel Selection} is an iterative algorithm, going for each $w_i \in G$ and for each channel $c \in w_i.channels$ running $OnIfaceInterference(G, w_i)$ until there is no improvement, so its worst-case time complexity can be estimated as $O(\card{G} \cdot C \cdot K \cdot \card{S_i})$, where $C$ is the number of channels and $K$ is the number of iterations;
    \item \textbf{Adjust Transmission Power} is an iterative algorithm, going for each $w_i \in G$ until there is no improvement yielded by $FromIfaceInterference()$, so its worst-case time complexity can be estimated as $O(\card{G} \cdot K \cdot \card{G}) = O({\card{G}}^2 \cdot K)$, where $K$ is the number of iterations;
    \item \textbf{Final Data Assembly and Event Generation} has complexity $O(\card{G})$.
\end{enumerate}

\subsection{RRMGreedy: limitations and conclusions}
\label{sec:flaws}

Based on the analysis carried out above in this chapter, we can identify the following issues with the \textit{RRMGreedy} algorithm:
\begin{enumerate}
    \item \textit{ChannelInterference()}, calculating how much two channels interfere, only yields binary values 1 or 0. Due to the presence of cross-channel interference, the result of \textit{ChannelInterference()} should be continuous in range $[0; 1]$, not discrete;
    % \item Following on the previous point, interference estimation is not accurate. It is based on signal power only.
    \item Noise floor should be included into the interference metric, since it can drastically affect the performance on the channel;
    \item Channel utilization is not taken into account. Consider the case when 3 APs with no associated stations operate on channel $A$, and 1 AP with 10 associated stations operates on channel $B$. The algorithm does not keep track of the number of stations or other channel utilization data, and will favor $B$, even though $A$ is less utilized, since the only traffic APs on $A$ are likely to produce is rare and periodic beacon frames;
    \item Assuming maximum transmit power for each in-group AP in the Step 1 of Algorithm \ref{alg:greedyRRM} is not reasonable, since it tends to exaggerate the interference scores and thus lead to unnecessary channel switches;
    \item When calculating the interference score, if there are several signals from the same APs, they will be summed as they were different APs, again leading to exaggerated interference scores;
    \item Even if next iteration yields worse interference score, there is no rollback to the previous, better state;
    \item Change with no improvements will be staged anyway, so disruptive channel switch will occur even if it is not necessary;
    \item Finally, selecting channel that minimizes the interference for an interface in a greedy way could not lead to the global optimum.
    % \item TPC sums all historical record for TPC. Why, if the latest record is the most adequate one
\end{enumerate}

% \subsection{Evaluation on real-world data}
% \label{sec:eval}
\section{RRMGreedy++ design}
\label{chap:research:sec:rrmv2}
% As discussed in \ref{chap:lr:sec:rrm_approaches}, RRM algorithm, given the current state of network, that is, who can hear who and how strong, should decide on the channel and transmit power assignments for every AP, so that network capacity and throughput are maximized. For per-cell decisions, it is simple, and most of the kinds of the algorithm can fit into LCCS definition: scan each channel, get an estimate how good each channel is (using a certain metric, such as interference, number of APs operating on the same channel, etc.), and switch to the best one. On the other hand, a super-cell aims to make decisions improving the performance of the whole network. Usually this is achieved by using a cumulative metric across all cells, such as total interference, and trying to minimize it. Thus, a channel and (or) transmit power allocation that minimizes the cumulative metric should be chosen. As shown above, RRMGreedy selects channels to minimize interference at each AP, while there is an improvement cumulative metric.

Based on the limitations identified in Section \ref{sec:flaws}, I introduce the following improvements:

\begin{enumerate}
    \item \textbf{Continuous interference estimation}: I propose a slightly updated \textit{ChannelInterference} implementation, referred to as \textit{ChannelInterference++}, that estimates interference score based on how much do channels overlap (Algorithm \ref{lst:channel_interfv2});
    \item \textbf{Updated cumulative metrics approach}:
        \begin{itemize}
            \item \textbf{Noise floor} for \textit{OnIfaceInterference()} cumulative estimation. This affects channel selection decision, discouraging channels with high noise floor level;
            \item \textbf{Channel utilization}: cumulative interference metric now takes into account channel utilization in its simplest form: number of clients of the AP. This will discourage APs from using channels with many active clients that can be heard.
        \end{itemize}
        These improvements are used in \textit{OnIfaceInterference++} (Algorithm \ref{lst:onifaceinterfv2}). I also refactored the program to make it more simpler while preserving the behaviour. Namely, splitting signals into "inner" and "outer" ones is not reasonable. The only reason why such division can take place is \textit{TxDiff} adjustment for inner devices, and this is trivial to implement within one loop.
    \item \textbf{RRMGreedy Flow Improvements}:
        \begin{itemize}
            \item The "$TxDiff \rightarrow \max$" phase of RRMGreedy (Step 1 of Algorithm \ref{alg:greedyRRM}) is moved after the Channel Selection part and before the Transmission Power Adjustment part;
            \item The \textit{GroupState} structure, containing the current state of \rrmG, is saved for the previous iteration of the algorithm. If the interference score increases, \textit{RRMGreedy++} aborts and returns to the previous \textit{GroupState};
            \item Iterations that yield a new group configuration with $GroupInterference < \epsilon$, where $\epsilon$ is a configurable threshold pre-defined to $\epsilon = 0.005$, are discarded.
        \end{itemize}
\end{enumerate}

\begin{algorithm}[H]
\caption{ChannelInterference++}
\label{lst:channel_interfv2}
\DontPrintSemicolon % Don't print semicolons at the end of each line
\SetKwFunction{FMain}{ChannelInterferencePlusPlus}
\SetKwProg{Fn}{Function}{:}{}
\Fn{\FMain{$ch_1$, $ch_2$, $width = 20$}}{
    \If{$ch_1 < 36$ and $ch_2 < 36$}{ % 2.4 GHz band
        $freqDiff \gets \card{ch_1 - ch_2} \times 5$\;
        $maxDistance \gets width + 5$\; % Maximum distance for full interference includes channel width plus adjacent overlap
    }
    \ElseIf{$ch_1 \geq 36$ and $ch_2 \geq 36$}{ % 5 GHz band
        $freqDiff \gets \card{ch_1 - ch_2} \times 20$\;
        $maxDistance \gets width$\; % In 5 GHz band, max distance is the channel width itself
    }
    \If{$freqDiff > maxDistance$}{
        \Return 0.0\; % No significant interference
    }
    $overlap \gets 1 - \frac{freqDiff}{maxDistance}$\; % Linear interpolation between 0 and 1 based on overlap
    \Return $overlap$\;
}
\end{algorithm}

\begin{algorithm}[H]
\label{lst:onifaceinterfv2}
\caption{OnIfaceInterference++}
\DontPrintSemicolon
\KwIn{
    $groupState$: table containing state of \rrmG \newline
    $interface$: interface to calculate interference for
    $channel$: channel to calculate interference on
}
\KwOut{Estimated cumulative interference score experienced by $interface$}

$ifaceScanData \gets scandata[interface]$\;
$cumInterf \gets 0$\;

\tcp{Calculate interference from both inner and outer APs}
\ForEach{otherBssid, otherSignal in groupState[interface].receivedSignals}{
    $ciScore \gets ChannelInterference(otherSignal.RSSI, channel, interface.ChannelWidth)$\;
    \ForEach{signal in otherChannelSignals}{
        $sigRating \gets otherSignal.RSSI$\;
        \If{otherBssid $\in$ groupState}{ \tcp{inner interface}
            $sigRating \gets sigRating + groupState[otherBssid].TxDiff$\;
        }
        \tcp{$sigRating$ (in dBm) is normalized to $[0, 1]$ scale}
        $sigRating \gets scale(sigRating)$\;
        $cumInterf \gets ciScore \times (sigRating + clientsWeight\times clients[bssid].len - otherSignal.noiseFloor)$\;
    }
}

\Return $cumInterf$\;
\end{algorithm}

\renewcommand{\algorithmcfname}{ALGORITHM}
\SetAlFnt{\small}
\SetAlCapFnt{\small}
\SetAlCapNameFnt{\small}
\LinesNumbered
\begin{algorithm}[H]
\caption{RRMGreedy++}
\label{alg:greedyRRMv2}

\SetKwInput{KwInput}{Input}
\SetKwInput{KwOutput}{Output}
\SetKwInput{KwData}{Data}
\SetKwComment{Comment}{$\triangleright$\ }{}
\SetKwFor{For}{for}{do}{endfor}
\SetKwIF{If}{ElseIf}{Else}{if}{then}{else if}{else}{endif}
\SetKwRepeat{Repeat}{repeat}{until}

\DontPrintSemicolon

\KwInput{
    $\boldsymbol{scandata}$: scanning data from access points \newline
    $\boldsymbol{G}$: set of wireless interfaces within the RRM group and its settings
}
\KwOutput{$\boldsymbol{resultConfig}$: final configuration for access points}

\BlankLine
groupState $\gets$ PreprocessRRMData($\boldsymbol{G}$, scandata)\;
\BlankLine
\tcp{Step 1: Greedy Channel Search}
initialGroupInterference $\gets$ nil\;

groupInterference $\gets$ GroupInterference(groupState)\;
\Repeat{no improvement in groupInterference}{
    \For{each interface in $\boldsymbol{G}$}{
        $\boldsymbol{C_i} \gets$ available channels for interface\;
        newChannel $\gets \argmin_{ch \in \boldsymbol{C_i}} OnIfaceInterferencePlusPlus(groupState, interface, ch)$\;
        groupState[interface].channel $\gets$ newChannel\;
    }
    groupInterference $\gets$ GroupInterference(scandata)\;
}
\BlankLine
\tcp{Step 2: Assume max Tx power for each interface}
\For{each interface in $\boldsymbol{G}$}{
    groupState[interface].TxDiff $\gets (interface.MaxTxPower - interface.TxPower)$\;
}
\BlankLine
\tcp{Step 3: Greedy Transmission Power Adjustment (if enabled)}
\Repeat{no improvement in groupInterference}{
    \tcp{Identify the "worst interface" $interface_{worst}$ causing the greatest interference for other APs\;
    $interface_{worst}$ $\gets \argmax_{w_i \in G} FromIfaceInterference(groupState, w_i)$\;
    $groupState[interface_{worst}].TxDiff \gets \frac{interface_{worst}.TxPower - interface_{worst}.MinTxPower}{2}$ \;
    groupInterference $\gets$ GroupInterference(scandata)\;
}
}
\BlankLine
\tcp{Step 4: Send configuration changes to APs}
$\boldsymbol{resultConfig} \gets$ getUpdatedConfiguration(groupState)\;
\Return{$\boldsymbol{resultConfig}$}\;
\end{algorithm}


\section{Conclusion}
\label{chap:research:sec:conclusion}
In this chapter, I have performed an analysis of a production RRM algorithm called \textit{RRMGreedy}, used by a leading Russian telecommunications vendor. By analyzing the source code of \textit{RRMGreedy}, I have estimated the asymptotic complexity and found several issues potentially affecting the efficiency of the algorithm. To address those issues, I propose a set of enhancements for \textit{RRMGreedy}, together called \textit{RRMGreedy++}. In Chapter \ref{chap:eval}, I proceed with experimental evaluation of the RRM algorithms, including \textit{RRMGreedy++}, by introducing a new framework for ns-3 network simulator I have created to facilitate RRM assessment.
