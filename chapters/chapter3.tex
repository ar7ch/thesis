\definecolor{codegreen}{rgb}{0,0.6,0}
\definecolor{codegray}{rgb}{0.5,0.5,0.5}
\definecolor{codepurple}{rgb}{0.58,0,0.82}
\definecolor{backcolour}{rgb}{0.95,0.95,0.92}

\lstdefinestyle{mystyle}{
    backgroundcolor=\color{backcolour},   
    commentstyle=\color{codegreen},
    keywordstyle=\color{magenta},
    numberstyle=\tiny\color{codegray},
    stringstyle=\color{codepurple},
    basicstyle=\ttfamily\footnotesize,
    breakatwhitespace=false,         
    breaklines=true,                 
    captionpos=b,                    
    keepspaces=true,                 
    numbers=left,                    
    numbersep=5pt,                  
    showspaces=false,                
    showstringspaces=false,
    showtabs=false,                  
    tabsize=2
}

\lstdefinelanguage{Lua}
{
morekeywords={and,break,do,else,elseif,end,false,for,function,if,in,local,nil,not,or,repeat,return,then,true,until,while,_G,_ENV,table,every},
sensitive=false,
morecomment=[l]{--},
morecomment=[s]{--[[}{]]},
morestring=[b]",
morestring=[b]'
}

\lstset{style=mystyle}


\chapter{Problem statement and baseline solution analysis}
\label{chap:met}


The purpose of this chapter is to lay a foundation for the further work in this Thesis via formal definition of RRM problem and analysis of existing RRM algorithm from Wimark Systems. The chapter is organized as follows. Section \ref{sec:prob} defines the problem of RRM and its goals. Section \ref{sec:baseline} describes the baseline RRM algorithm from Wimark Systems and considerations that led to such solution.


\section{Problem statement}
\label{sec:prob}


\section{Baseline solution analysis}
\label{sec:baseline}
The legacy RRM algorithm from Wimark Systems that we will refer to as \textit{Legacy} is based on background scanning by access points. The goal of this thesis is to create a new RRM algorithm that is overall better in managing radio resources and at least not worse than \textit{Legacy} in corner cases.

Here, we describe \textit{Legacy}.
First

\begin{lstlisting}[language={[5.0]Lua}, caption=Legacy RRM algorithm, label=lst:legacy]

    function backgroundScan():
        for each band:
            for each channel:
                sniff frames (only beacon?) on channel
                collect SSIDs and corresponding RSSI
        return table {SSID, Band, Channel, RSSI}

    function collectRRMStats():
        for each AP with background scanning enabled:
            backscan := background_scan()
            send backscan results to WLC DB

    function ap_adjust(ap, powerAssignment, channelAssignment):
        ap.setChannel(channelAssignment)
        ap.setPower(powerAssignment)

    every 10 minutes on WLC:
        rrmStats = collectRRMStats()
        upload rrmStats to WLC DB


    every RRM_PERIOD minutes or on demand:
        rrmStats = load rrmStats from WLC DB
        for each rfGroup in RFGroups:
            channelAssignments, powerAssignments = RRMAlgorithm(rfGroup)
            for each ap in rfGroup:
                ap_adjust(ap, channelAssignments[ap], powerAssignments[ap])


    function RRMAlgorithm(rfGroup):
        ScanData = {SSID, BSSID, RSSI, Channel, }
        RRMData = {APSettings, APModelInfo, ScanData}
        powerAssignments = decideTPC(RRMData)
        channelAssignments = decideACS(RRMData)
        return channelAssignments, powerAssignments

    function decideTPC(RRMData):
        powerAssignments = {}
        for each ap in RRMData:
            powerAssignments[ap] = maxPossiblePower(ap)
        return powerAssignments

    function decideACS(RRMData):
        currentGroupCumInterference = cumInteference(RRMData)

    function cumInteference(RRMData):
        groupInterference = 0
        for each ap in RRMData:
            for each interface on ap:
                groupInterference += interference(interface.channel, interface.width)
        return groupInterference

    function interference(ifaceChannel, ifaceWidth):
        for each otherChannel in otherChannels:
            ci = channelInterference(ifaceChannel, ifaceWidth, otherChannel)
            if ci == 0:
                continue
            


\end{lstlisting}

