\begin{abstract}
As IEEE 802.11-compatible ("Wi-Fi") Wireless Local Area Networks (WLAN) become ubiquitous, the rising density of WLAN deployments leads to congestion of frequency bands and performance degradation due to interference. Thus, Radio Resource Management (RRM) algorithms for mitigating those problems become in demand, especially for large WLAN deployments.

While several commercial RRM solutions exist, their implementation details remain undisclosed. Moreover, such solutions are incompatible between devices produced by different vendors. At the same time, most of the previous studies have considerable obstacles for production usage.

In this study, I carry out theoretical analysis of RRMGreedy, a greedy centralized RRM algorithm used by a major Russian telecommunications vendor. I show that RRMGreedy possess a number of flaws and tends to yield suboptimal results.
After identifying flaws of RRMGreedy, I propose a novel centralized RRM algorithm for multi-AP WLAN deployments, called RRMv2. This algorithm adjusts frequency channel and transmission power parameters of access points based on physical-layer metrics, and uses WLAN Controller (WLC) as a central entity to gather data and perform computations.

Additionally, this study develops a novel framework for the ns-3 network simulator that is used to implement RRMv2 alongside with other RRM algorithms.
Simulation results demonstrate that RRMv2 achieves up to a 17\% faster convergence rate on 2.4 GHz IEEE 802.11n networks and improves Signal-to-Interference-Plus-Noise Ratio (SINR) by 31\% and WLAN throughput by up to 29\% over previous RRM methods.

The RRMv2 algorithm employs unsophisticated metrics that can be obtained from the majority of OpenWRT-based Wi-Fi access points available on the market, demonstrating that the algorithm is production-ready and can be easily integrated into existing enterprise WLAN infrastructures.
\end{abstract}