\appendix

\makeatletter
\let\@currsize\normalsize
\makeatother
\label{appx:defs_table}
\begin{table}[h]
    \centering
    \begin{tabular}{| p{0.5\linewidth} | p{0.5\linewidth} |}\hline
        \textbf{Term} & \textbf{Definition} \\ \hline
        Signal & Airborne RF energy \\
        \hline
        Channel & A band of frequencies that 802.11 devices can use for communications \cite{AuthoritativeDictionaryIEEE2000} \\
        \hline
        Inteference & Destructive influence of another signal leading to degradation of signal quality and loss of frames \\
        \hline
        Noise & Signal that cannot be demodulated as 802.11 signal \\
        \hline
        SNR (Signal-to-Noise Ratio) & Signal quality metric, defined as ratio of signal power to the noise \\
        \hline
        SINR (Signal-to-Interference-plus-Noise Ratio) & Signal quality metric, defined as ratio of signal power to the sum of noise power and power of interefering signals \\
        \hline
        RSSI (Received Signal Strength Indicator) & A metric of a wireless signal quality defined in 802.11, varying from 0 to 255, where exact mappings to Rx signal power are vendor-specific. \\
        \hline
        Radio cell & Geographical area covered by a radio transmitter \cite{tanenbaumComputerNetworks2020} \\
        \hline
        Basic Service Set (BSS) & Set of stations belonging to the same radio cell and exchanging information \cite{konsgenSpectrumManagementAlgorithms2010} \\
        \hline
        Distribution System (DS) & Network that interconnects multiple BSSs and provides connectivity to a wired network \cite{konsgenSpectrumManagementAlgorithms2010} \\
        \hline
        Extended Service Set (ESS) & Set of Basic Service Sets connected by a DS \cite{konsgenSpectrumManagementAlgorithms2010} \\
        \hline
        Station (STA) & Client device that can connect to a wireless LAN \\
        \hline
        Access Point (AP) & Device providing stations wireless access to a distribution system \\
        \hline
        Infrastructure mode & Centralized mode of 802.11 WLAN operation using a star topology, where AP serves as a central entity for managing WLAN and switching traffic \\
        \hline
        Network capacity & Maximum transmission rate that any station can achieve in a given WLAN \\
        \hline
    \end{tabular}
    \caption{Used terms and definitions}
    \label{tab:my_label}
\end{table}
% \chapter{Extra Stuff}
% \blindtext

% \chapter{Even More Extra Stuff}
% \blindtext