\chapter{Conclusion}
\label{chap:conclusion}

In this thesis, I developed and evaluated an improved the Radio Resource Management (RRM) algorithm, \textit{RRMGreedy++}, based on \textit{RRMGreedy}. The study identified the limitations of \textit{RRMGreedy} and introduced \textit{RRMGreedy++}, leading to a 12\% improvement in bandwidth and a 10\% increase in signal-to-noise ratio. 

Significant contributions include:
1. Identification and improvement of \textit{RRMGreedy} flaws.
2. Development of a novel RRM framework in ns-3.
3. Enhanced performance of \textit{RRMGreedy++} in simulations.

Limitations of the study, such as the constraints of the \texttt{YansWifiPhy} propagation model, suggest areas for further research. Future work should focus on:
1. Enhancing the simulation model to better reflect real-world conditions.
2. Exploring alternative channel selection methods, such as genetic algorithms.
3. Implementing and testing \textit{RRMGreedy++} in real-world environments.

In summary, \textit{RRMGreedy++} demonstrates significant potential for improving radio resource utilization, with future work aimed at further refinement and real-world validation.

